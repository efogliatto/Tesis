\begin{resumen}
En la presente tesis se analiz\'o la expansi\'on de la t\'ecnica hacia la resoluci\'on del transporte de energ\'ia en flujos multif\'asicos, buscando un camino para resolver computacionalmente, de manera eficiente, el elusivo fen\'omeno de ebullici\'on. Para ello, se abord\'o el camino propuesto siguiendo una metodolog\'ia incremental. En primer lugar, se comenz\'o con la evaluaci\'on de modelos multif\'asicos isot\'ermicos, compararando la capacidad de reproducir problemas con soluci\'on anal\'itica. Estos modelos fueron complementados con nuevas propuestas para resolver el transporte de energ\'ia en dos y tres dimensiones, junto con una metodolog\'ia de an\'alisis consistente y necesaria para esta t\'ecnica. Finalmente, los nuevos modelos y procedimientos fueron validados a partir de la simulaci\'on de un problema de ebullici\'on, y la comparaci\'on de resultados con mediciones experimentales. Los modelos LB utilizados fueron implementados en una herramienta num\'erica desarrollada en C++, que permite efectuar las simulaciones en arquitecturas de alto desempe\~no.
\end{resumen}

\begin{abstract}
En la presente tesis se analiz\'o la expansi\'on de la t\'ecnica hacia la resoluci\'on del transporte de energ\'ia en flujos multif\'asicos, buscando un camino para resolver computacionalmente, de manera eficiente, el elusivo fen\'omeno de ebullici\'on. Para ello, se abord\'o el camino propuesto siguiendo una metodolog\'ia incremental. En primer lugar, se comenz\'o con la evaluaci\'on de modelos multif\'asicos isot\'ermicos, compararando la capacidad de reproducir problemas con soluci\'on anal\'itica. Estos modelos fueron complementados con nuevas propuestas para resolver el transporte de energ\'ia en dos y tres dimensiones, junto con una metodolog\'ia de an\'alisis consistente y necesaria para esta t\'ecnica. Finalmente, los nuevos modelos y procedimientos fueron validados a partir de la simulaci\'on de un problema de ebullici\'on, y la comparaci\'on de resultados con mediciones experimentales. Los modelos LB utilizados fueron implementados en una herramienta num\'erica desarrollada en C++, que permite efectuar las simulaciones en arquitecturas de alto desempe\~no.
\end{abstract}
