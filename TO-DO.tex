- En 3.2, mencionar por qu\'e se llama pseudopotencial
- Uniformizar nomenclatura: _i para nodo i-esimo, _{\alpha} para componente {\alpha-}esima
- por qu\'e p = \rho c_s^2 + G/ggdg es una ecuaci\'on de estado. Hay que poner en 3.2 cu\'al es la ecuaci\'on de momento que se recupera
- Ver articulo de Shan 2008 por balance discreto, y c\'omo se llega al tensor de presi\'on.
- Cambiar w por omega en los pesos de la funcion de equilibrio del cap\'itulo 2
- Nombres en ingles de los modelos multif\'asicos: free-energy, color-gradient, etc. Quedan as\'i, o hay que traducirlos?