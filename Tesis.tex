%%%%%%%%%%%%%%%%%%%%%%%%%%%%%%%%%%%%%%%%%%%%%%%%%%%%%%%%%%%%%%%%%%%%%%%%%%%%%%%%
% \documentclass[12pt,papel,twoside]{ibtesis}
\documentclass[12pt,screen,twoside,pagebackref]{IBTesis}
%\documentclass[12pt,screen,twoside,pagebackref]{IBTesis}
% \documentclass[12pt,papel,singlespace,oneside]{ibtesis}
% \documentclass[12pt,papel,preprint,singlespace,oneside]{ibtesis}


%%%%%%%%%%%%%%%%%%%%% Paquetes extra %%%%%%%%%%%%%%%%%%%%%%%%%%%%%%%%%%%%%%%%%%%
% Por conveniencia: aqu\'{\i} puede cargar todos los paquetes y definir los comandos 
% que necesite
\usepackage{IBExtra}
%%%%%%%%%%%%%%%%%%%%%%%%%%%%%%%%%%%%%%%%%%%%%%%%%%%%%%%%%%%%%%%%%%%%%%%%%%%%%%%%
%%%%%%%%%%%%%%%%%%%%% Informacion sobre la tesis %%%%%%%%%%%%%%%%%%%%%%%%%%%%%%%
\title{Simulaci\'on num\'erica del fen\'omeno de ebullici\'on empleando el m\'etodo de lattice Boltzmann}
\author{Ezequiel O. Fogliatto}
\director{Dr.~Federico~E.~Teruel}
\codirector{Dr.~Alejandro~Clausse}
\carrera{Tesis Carrera de Doctorado en Ciencias de la Ingenier\'ia}
\grado{Doctorando}
\laboratorio{Departamento de Mec\'anica Computacional -- Centro At\'omico Bariloche}
\jurado{Dr.~J.~J.~Jurado (Instituto Balseiro) \\ 
Dr.~Segundo Jurado (Universidad Nacional de Cuyo)\\ 
Dr.~J.~Otro Jurado (Univ. Nac. de LaCalle)\\
Dr.~J.~L\'{o}pez Jurado (Univ. Nac. de Mar del Plata)\\
Dr.~U.~Amigo (Instituto Balseiro, Centro At\'{o}mico Bariloche)}
\palabrasclave{formato de Tesis, Lineamientos de escritura, Instituto Balseiro}
\keywords{Thesis format, Templates, Instituto Balseiro}
% Si queremos poner la fecha manualmente:
% \date{Diciembre de 2099}

%%%%%%%%%%%%%%%%%%%%%%%%%%%%%%%%%%%%%%%%%%%%%%%%%%%%%%%%%%%%%%%%%%%%%%%%%%%%%%%%
%\titlepagefalse % Si no quiere compilar la portada descomente esta linea
%\includeonly{apendices} % Compilar s\'{o}lo estos archivos 
\graphicspath{{figs/}} % Lugar donde encontrar las figuras generales (se puede poner uno en cada cap{\'{\i}}tulo)
%%%%%%%%%%%%%%%%%%%%%%%%%%%%%%%%%%%%%%%%%%%%%%%%%%%%%%%%%%%%%%%%%%%%%%%%%%%%%%%%


\begin{document}

% Dentro del environment 'preliminary' va:
% la dedicatoria, resumen, abstract, indices

\begin{preliminary}

% Escriba su dedicatoria
\dedicatoria{
A mi familia
}

%%% \'{I}ndices %%%%

\begin{abreviaturas}
                                %Abreviaturas
\end{abreviaturas}

\tableofcontents                %\'{I}ndice

\listoffigures                  %Figuras

\listoftables                   %Tablas

\include{resumen}

\end{preliminary}



\chapter{Introducci\'on}

Prueba de citas: \cite{fogliatto_simulation_2019}

\chapter{Fundamentos de lattice Boltzmann}

En este cap\'itulo se describir\'an los fundamentos necesarios y la sarasa obligatoria para m\'as o menos entender el detalle de un modelo de lattice Boltzmann.
Poner ac\'a la idea de mostrar este camino para llegar a lo que nos interesa de LB. Se puede comenzar mencionando el origen como aut\'omatas celulares, y el posterior descubrimiento como forma discreta de la ecuaci\'on de Boltzmann. En definitiva, \'esta \'ultima es la que abre el camino a usarlo como m\'etodo de resoluci\'on de PDE's. Ver r\'apido en Huang, Sukop, Lu.

\section{Naturaleza cin\'etica del m\'etodo}
\label{sec:kinetic}
La descripci\'on matem\'atica de la din\'amica de fluidos se basa en la hip\'otesis de un medio continuo, con escalas temporales y espaciales suficientemente mayores que las asociadas a la naturaleza atom\'istica subyacente. En este contexto, suelen encontrarse referencias a descripciones microsc\'opicas, mesosc\'opicas o macrosc\'opicas. La descripci\'on microsc\'opica, por un lado, hace referencia a una descripci\'on molecular, mientras que la macrosc\'opica involucra una visi\'on continua completa, con cantidades tangibles como densidad o velocidad del fluido. Por otro lado, entre ambas aproximaciones se encuentra la teor\'ia cin\'etica mesosc\'opica, la cu\'al no describe el movimiento de part\'iculas individuales, sino de distribuciones o colecciones representativas de dichas part\'iculas.
\par
La variable fundamental de la teor\'ia cin\'etica se conoce como funci\'on de distribuci\'on de part\'iculas (\emph{particle distribution function}, o pdf por sus siglas en ingl\'es), que puede verse como una generalizaci\'on de la densidad $\rho$ y que a su vez tiene en cuenta la velocidad microsc\'opica de las part\'iculas $\bm{\xi}$. Por lo tanto, mientras que $\rho(\bm{x},t)$ representa la densidad de masa en el espacio f\'isico, la funci\'on de distribuci\'on \fvar{} corresponde a la densidad de masa tanto en el espacio f\'isico como en el espacio de velocidades.
\par
La funci\'on de distribuci\'on $f$ se relaciona con variables macrosc\'opicas como densidad $\rho$ y velocidad $\bm{u}$ a trav\'es de momentos, es decir, integrales de $f$ con funciones de peso dependientes de \bxi{} sobre todo el espacio de velocidades. En particular, la densidad de masa macrosc\'opica puede obtenerse como el momento

\begin{equation}
	\rho(\bm{x},t) = \int f(\bm{x},\bm{\xi},t) \, d^3 \xi,
\end{equation}
en el cual se considera la contribuci\'on de part\'iculas con todas las velocidades posibles en la posici\'on $\bm{x}$ a tiempo $t$. Por otro lado, puede determinarse la densidad de impulso mediante
\begin{equation}
	\rho(\bm{x},t) \bm{u}(\bm{x},t) = \int \bm{\xi} f(\bm{x},\bm{\xi},t) \, d^3 \xi.
\end{equation}

De forma similar, la densidad de energ\'ia total corresponde al momento
\begin{equation}
	\rho(\bm{x},t) E(\bm{x},t) = \dfrac{1}{2} \int |\bm{\xi}|^2 f(\bm{x},\bm{\xi},t) \, d^3 \xi.
\end{equation}


\subsection{Funci\'on de distribuci\'on de equilibrio}
En el an\'alisis original realizado para gases iluidos y monoat\'omicos, Maxwell menciona que cuando un gas permanece sin perturbaciones por un per\'iodo de tiempo suficientemente largo, la funci\'on de distribuci\'on \fvar{} alcanza una distribuci\'on de equilibrio \feqvar{} que es isotr\'opica en el espacio de velocidades en torno a $\bm{\xi} = \bm{u}$. De esta manera, si te toma un marco de referencia que se desplaza con velocidad $\bm{u}$, entonces dicha distribuci\'on de equilibrio puede expresarse como $f^{eq}(\bm{x},|\bm{v}|,t)$. Por otro lado, si se supone que la distribuci\'on de equilibrio puede expresarse de forma separable, es decir 

\begin{equation}
	f^{eq}(|\bm{v}|^2) = f^{eq}(v_x^2 + v_y^2 + v_z^2)=f_{1D}^{eq}(v_x^2) \, f_{1D}^{eq}(v_y^2) \, f_{1D}^{eq}(v_z^2),
\end{equation}
entonces puede demostrarse que dicha distribuci\'on queda definida como

\begin{equation}
	f^{eq}(\bm{x},|\bm{v}|^2,t) = \mbox{e}^{3a}\mbox{e}^{b|\bm{v}|^2}.
\end{equation}
Por otro lado, considerando que las colisiones monoa\'omicas conservan masa, momento y energ\'ia, y usando adem\'as la relaci\'on de gases ideales:
\begin{equation}
	\rho e = \frac{3}{2}RT=\frac{3}{2}p,
\end{equation}
finalmente puede encontrarse una expresi\'on expl\'icita para la distribuci\'on de equilibrio
\begin{equation}
	f^{eq}(\bm{x},|\bm{v}|,t) 
	= \rho \left( \dfrac{3}{4\pi e} \right)^{3/2} \mbox{e}^{-3|\bm{v}|^2/(4e)}
	= \rho \left( \dfrac{1}{2\pi RT} \right)^{3/2} \mbox{e}^{-|\bm{v}|^2/(2RT)}
\end{equation}

\subsection{La ecuaci\'on de Boltzmann}
La funci\'on de distribuci\'on \fvar{} establece propiedades tangibles de un fluido a trav\'es de sus diferentes momentos. Asimismo, es posible determinar una ecuaci\'on que permita modelar su evoluci\'on en el espacio f\'isico, de velocidades, y el tiempo. En el an\'alisis siguiente, se omitir\'a la dependencia de $f$ con $(\bm{x}, \bm{\xi}, t)$ por claridad.
\par
Como $f$ es una funci\'on de la posici\'on $\bm{x}$, de la velocidad de las part\'iculas \bxi{}, y del tiempo $t$, la derivada total respecto al tiempo resulta

\begin{equation}
	\dfrac{df}{dt} = \left( \dfrac{\partial f}{\partial t} \right) \dfrac{dt}{dt}
	               + \left( \dfrac{\partial f}{\partial x_{\beta}} \right) \dfrac{dx_{\beta}}{dt}
	               + \left( \dfrac{\partial f}{\partial \xi_{\beta}} \right) \dfrac{d\xi_{\beta}}{dt}.
\end{equation}

En este caso $dt/dt = 1$, la velocidad de las part\'iculas se obtiene como $dx_{\beta}/dt = \xi_{\beta}$, y la fuerza volum\'etrica $\bm{F}$ queda determinada por la segunda ley de Newton $d\xi_{\beta}/dt = F_{\beta}/\rho$. Utilizando la notaci\'on tradicional $\Omega (f) = df/dt$ para el diferencial total respecto al tiempo, se obtiene la ecuaci\'on de Boltzmann para describir la evoluci\'on de $f$:

\begin{equation}
	\dfrac{\partial f}{\partial t}  +  \xi_{\beta} \dfrac{\partial f}{\partial x_{\beta}}            +  \dfrac{F_{\beta}}{\rho} \dfrac{\partial f}{\partial \xi_{\beta}} =\Omega(f).
	\label{eq:boltz}
\end{equation}

La \eq{eq:boltz} puede verse como una ecuaci\'on de advecci\'on para $f$, donde los dos primeros t\'erminos del miembro izquierdo corresponden a la advecci\'on de $f$ con la velocidad de part\'iculas \bxi{}, mientras que el tercero representa el efecto de las fuerzas externas. Por otro lado, el miembro derecho contiene un t\'ermino de fuente conocido como operador de colisi\'on, que representa la redistribuci\'on local de $f$ debido a colisiones entre las propias part\'iculas. Estas colisiones conservan masa, momento y energ\'ia, lo que se traduce en restricciones para los momentos de $\Omega$:
\begin{subequations}
	\begin{align}
		\int \Omega(f) \, d^3 \xi = 0 \\
		\int \bm{\xi} \, \Omega(f) \, d^3 \xi = \bm{0} \\		
		\int |\bm{\xi}|^2 \, \Omega(f) \, d^3 \xi = 0 	
	\end{align}
	\label{eq:omega_restrict}
\end{subequations}


\subsection{Ecuaciones de conservaci\'on macrosc\'opicas}
Las ecuaciones de conservaci\'on macrosc\'opicas pueden obtenerse como momentos de la ecuaci\'on de Boltzmann, es decir, multiplicando la \eq{eq:boltz} por funciones de \bxi{} e integrando sobre todo el espacio de velocidades. Para ello, es necesario introducir una notaci\'on general para los momentos de $f$

\begin{subequations}
	\begin{equation}
		\Pi_0 = \int f \, d^3 \xi = \rho
	\end{equation}
	\begin{equation}
		\Pi_{\alpha} = \int \xi_{\alpha} f \, d^3 \xi = \rho u_{\alpha}
	\end{equation}
	\begin{equation}
		\Pi_{\alpha \beta} = \int \xi_{\alpha} \xi_{\beta} f \, d^3 \xi
	\end{equation}
	\begin{equation}
		\Pi_{\alpha \beta \gamma} = \int \xi_{\alpha} \xi_{\beta} \xi_{\gamma} f \, d^3 \xi
	\end{equation}
	\label{eq:f_moments}
\end{subequations}

La ecuaci\'on m\'as simple de obtener corresponde a la de conservaci\'on de masa. Integrando la \eq{eq:boltz} en el espacio de velocidades, y usando las \eqs{eq:omega_restrict}{eq:f_moments}, se obtiene:

\begin{equation}
	\dfrac{\partial \rho}{\partial t} + \dfrac{\partial (\rho u_{\beta})}{\partial x_{\beta}} = 0.
\end{equation}

De manera similar, multiplicando la \eq{eq:boltz} por $\xi_{\alpha}$ e integrando en el espacio de velocidades se obtiene la ecuaci\'on de conservaci\'on de momento:

\begin{equation}
	\dfrac{\partial \rho u_{\alpha}}{\partial t} + \dfrac{\partial \Pi_{\alpha \beta}}{\partial x_{\beta}} = F_{\alpha}.
	\label{eq:mom_flux}
\end{equation}
donde $\Pi_{\alpha \beta}$ se define como el tensor de flujo de impulso. Si se descompone la velocidad de las part\'iculas como $\bm{xi} = \bm{u} + \bm{v}$, entonces la \eq{eq:mom_flux} puede reescribirse como

\begin{equation}
	\dfrac{\partial \rho u_{\alpha}}{\partial t} 
	+ \dfrac{\partial (\rho u_{\alpha} u_{\beta})}{\partial x_{\beta}} = 
	\dfrac{\partial \sigma_{\alpha \beta}}{\partial x_{\beta}} + F_{\alpha}.
	\label{eq:impulso}
\end{equation}
con $\sigma_{\alpha \beta}$ representando el tensor de tensiones:

\begin{equation}
	\sigma_{\alpha \beta} = -\int v_{\alpha} v_{\beta} f \, d^3 \xi
\end{equation}

Finalmente, puede seguirse un procedimiento similar para encontrar una ecuaci\'on macrosc\'opica de conservaci\'on de energ\'ia. Multiplicando la \eq{eq:boltz} por $\xi_{\alpha} \xi_{\beta}$ e integrando en el espacio de velocidades se obtiene:

\begin{equation}
	\dfrac{\partial \rho E}{\partial t} + \dfrac{1}{2} \dfrac{\Pi_{\alpha\alpha\beta}}{\partial x_{\beta}} = F_{\beta} u_{\beta}.
	\label{eq:energy_flux}
\end{equation}

Descomponiendo el momento como en la ecuaci\'on de conservaci\'on de impulso y usando la \eq{eq:impulso} multiplicada por $u_{\alpha}$, la \eq{eq:energy_flux} puede reescribirse como:

\begin{equation}
	\dfrac{\partial \rho e}{\partial t} + \dfrac{(\rho u_{\beta} e)}{\partial x_{\beta}} = \sigma_{\alpha\beta} \dfrac{\partial u_{\alpha}}{\partial x_{\beta}} - \dfrac{\partial q_{\beta}}{\partial x_{\beta}},
\end{equation}
donde el flujo de calor $\bm{q}$ est\'a definido por el momento

\begin{equation}
	q_{\beta} = \dfrac{1}{2} \int v_{\alpha} v_{\alpha} v_{\beta} f \, d^3 \xi
\end{equation}


En este punto es interesante destacar que si bien la convervaci\'on de masa queda definida exactamente, las ecuaciones de impulso y energ\'ia dependen de la forma de $f$, que todav\'ia no es conocida. En el caso particular en que $f \simeq f^{eq}$, se obtienen las ecuaciones de Euler para impulso y energ\'ia:

\begin{subequations}
	\begin{equation}
	 	\dfrac{\partial \rho u_{\alpha}}{\partial t} 
	  + \dfrac{\partial (\rho u_{\alpha} u_{\beta})}{\partial x_{\beta}} 
	  =	-\dfrac{\partial p}{\partial x_{\alpha}} + F_{\alpha} 	
	\end{equation}
	\begin{equation}
		\dfrac{\partial \rho e}{\partial t} 
	  + \dfrac{(\rho u_{\beta} e)}{\partial x_{\beta}} 
	  = -p \dfrac{\partial u_{\beta}}{\partial x_{\beta}} 	
	\end{equation}
\end{subequations}

Este hecho muestra que los procesos macrosc\'opicos de disipaci\'on viscosa y difusi\'on de calor se encuentran directamente vinculados a la desviaci\'on de $f$ respecto de su valor de equilibrio.



\section{Discretizaci\'on del espacio de velocidades} 
El desarrollo mostrado en la \se{sec:kinetic} evidencia la posibilidad de representar adecuadamente el comportamiento de un fluido usando una funci\'on de distribuci\'on \fvar{}. Sin embargo, dicha distribuci\'on se encuentra definida en un espacio con 7 dimensiones, es decir, 3 coordenadas espaciales, 3 para el espacio de velocidades, y una para el tiempo, de modo que la resoluci\'on de ecuaciones en este espacio multidimensional involucra un esfuerzo computacional considerable. Por otro lado, es necesario considerar que este enfoque no es siempre justificable, dado que en definitiva son los momentos de la ecuaci\'on de Boltzmann (integrales en el espacio de velocidades) los que conducen a ecuaciones macrosc\'opicas de conservaci\'on de masa, impulso y energ\'ia. 
\par 
Estas caracter\'isticas originaron la b\'usqueda de versiones simplificadas de la ecuaci\'on de Boltzmann que no sacrifiquen el comportamiento macrosc\'opico, es decir, de sus momentos. Entre estas alternativas podemos encontrar las expansiones en base al n\'umero de Mach \cite{he_lattice_1997} o en series de Hermite \cite{shan_kinetic_2006}. Si bien ambas conducen a la misma representaci\'on de Navier-Stokes, la representaci\'on en series de Hermite presenta una base matem\'atica m\'as s\'olida, y es la que se utilizar\'a a continuaci\'on.
\par
La idea fundamental de la espansi\'on usando polinomios de Hermite consiste en simplificar la funci\'on de distribuci\'on de equilibrio $f^{eq}$ y discretizar el espacio de velocidades, pero manteniendo las leyes de conservaci\'on macrosc\'opicas. En particular, como $f^{eq}$ tiene una forma exponencial conocida, puede ser expresada a trav\'es de la funci\'on generatriz de dichos polinomios. Por otro lado, los momentos de masa e impulso son representados como integrales discretas de $f^{eq}$ usando los polinomios de Hermite.


\subsection{Adimensionalizaci\'on}
Antes de proceder con la discretizaci\'on de $f$ y $f^{eq}$ en series de Hermite, es conveniente reescribir las ecuaciones governantes de forma adimensional, con el objetivo de simplificar los pasos siguientes.
\par
La funci\'on de distribuci\'on \fvar{} representa la densidad de masa en el espacio f\'isico tridimensional y en el espacio de velocidades, tambi\'en tridimensional. Por lo tanto, las unidades de $f$ en el SI son:
\begin{equation}
	[f] = \mbox{kg} \times \dfrac{1}{\mbox{m}^3} \times \dfrac{1}{(\mbox{m/s})^3} = \dfrac{\mbox{kg} \, \mbox{s}^3}{\mbox{m}^6}.
\end{equation}

Las propiedades de un fluido pueden analizarse en t\'erminos de una longitud caracter\'istica $l$, velocidad caracter\'istica $V$ y densidad caracter\'istica $\rho_0$. Si se denota con * a las cantidades adimensionales, entonces podemos escribir los operadores diferenciales adimensionales como:
\begin{equation}
\dfrac{\partial}{\partial t^*} = \dfrac{l}{V}\dfrac{\partial}{\partial t}, \qquad
\dfrac{\partial}{\partial x^*} = l\dfrac{\partial}{\partial x}, \qquad
\dfrac{\partial}{\partial \xi^*} = V\dfrac{\partial}{\partial \xi}.
\end{equation}

Esto lleva a escribir a la forma adimensional de la ecuaci\'on de Boltzmann:
\begin{equation}
	\dfrac{\partial f^*}{\partial t^*}  
	+  \xi^*_{\alpha} \dfrac{\partial f^*}{\partial x^*_{\alpha}} 
	+  \dfrac{F^*_{\alpha}}{\rho^*} \dfrac{\partial f^*}{\partial \xi^*_{\alpha}} =\Omega^*(f^*),
	\label{eq:boltz_adim}
\end{equation}
donde $f^* = fV^d/\rho_0$, $\bm{F}^* = \bm{F}l/(\rho_0 V^2)$, $\rho^2 = \rho/\rho_0$ y $\Omega^*= \Omega l V^2 / \rho_0$. Siguiendo el mismo procedimiento, la funci\'on de equilibrio adimensional resulta:
\begin{equation}
	f^{eq*}	= \left( \dfrac{\rho^*}{2\pi \theta^*} \right)^{d/2} \mbox{e}^{-(\bm{\xi}^* - \bm{u}^*)^2/(2\theta^*)}
\end{equation}

En este caso, $\theta^*$ corresponde a la temperatura adimensional $\theta^*=RT/V^2$. En las secciones siguientes se trabajar\'a exclusivamente con cantidades adimensionales, omitiendo el super\'indice * por claridad.


\subsection{Expansi\'on en series de Hermite}
Las bases de la teor\'ia cin\'etica muestran que el operador de colisi\'on preserva ciertos momentos de la funci\'on de distribuci\'on, lo que a su vez implica que los momentos de $f^{eq}$ y $f$ deben coincidir:
\begin{subequations}
	\begin{align}
		\int f(\bm{x},\bm{\xi},t) \, d^3 \xi &=& \int f^{eq}(\rho, \bm{u},\theta,\bm{\xi}) \, d^3 \xi &=& \rho(\bm{x},t)\\
		\int f(\bm{x},\bm{\xi},t) \bm{xi}\, d^3 \xi &=& \int f^{eq}(\rho, \bm{u},\theta,\bm{\xi}) \bm{xi}\, d^3 \xi &=& \rho(\bm{x},t)\bm{u}(\bm{x},t)\\
		\int f(\bm{x},\bm{\xi},t) \dfrac{|\bm{\xi}|^2}{2} \, d^3 \xi &=& \int f^{eq}(\rho, \bm{u},\theta,\bm{\xi}) \dfrac{|\bm{\xi}|^2}{2} \, d^3 \xi &=& \rho(\bm{x},t)E(\bm{x},t)\\		
		\int f(\bm{x},\bm{\xi},t) \dfrac{|\bm{\xi}-\bm{u}|^2}{2} \, d^3 \xi &=& \int f^{eq}(\rho, \bm{u},\theta,\bm{\xi}) \dfrac{|\bm{\xi}-\bm{u}|^2}{2} \, d^3 \xi &=& \rho(\bm{x},t)e(\bm{x},t)		
	\end{align}
	\label{eq:cons_moments}	
\end{subequations}

Las cantidades conservadas de la \eq{eq:cons_moments} pueden obtenerse como integrales de $f$ o $f^{eq}$ en el espacio de velocidades. Por lo tanto, la expansi\'on en series de Hermite contribuye a transformar esas integrales continuas en sumas discretas evaluadas en puntos espec\'ificos del espacio de velocidades.
\par
Los polinomios de Hermite se definen en un espacio $d$-dimensional como: \cite{shan_kinetic_2006,grad_kinetic_1949}
\begin{equation}
	\bm{H}^{(n)} (\bm{x})= (-1)^n\dfrac{1}{\omega (\bm{x})} \bm{\nabla}^{(n)}\omega(\bm{x}),
\end{equation}
donde $\omega(\bm{x})$ es una funci\'on generatriz:
\begin{equation}
	\omega(\bm{x}) = \dfrac{1}{(2\pi)^{d/2}}\mbox{e}^{-\bm{x}^2/2}
\end{equation}

Tanto $\bm{H}^{(n)}$ como $\bm{\nabla}^{(n)}$ son tensores de rango $n$, de modo que sus $d^n$ componentes pueden expresarse como $H^{(n)}_{\alpha_1 \ldots \alpha_n}$ y $\nabla^{(n)}_{\alpha_1 \ldots \alpha_n}$, donde $\{\alpha_1 \ldots \alpha_n\}$ son \'indices comprendidos entre $1$ y $d$. Para el caso particular de una dimensi\'on, los polinomios se reducen a
\begin{equation}
	H^{(n)}(x)=(-1)^{(n)} \dfrac{1}{\omega(x)} \dfrac{d^n}{dx^n}\omega(x), \qquad 
	\omega(x) = \dfrac{1}{\sqrt{2\pi}}\mbox{e}^{-x^2/2}
\end{equation}

Los polinomios de Hermite son ortogonales respecto a la funci\'on de peso $\omega(\bm{x})$ y constituyen una base completa en $\Re^n$ \cite{wiener_fourier_1989}, de modo que es posible representar cualquier funci\'on $f(\bm{x})$ suficientemente suave mediante:
\begin{equation}
	f(\bm{x}) = \omega(\bm{x}) \sum_{n=0}^{\infty}\dfrac{1}{n!}\bm{a}^{(n)} \cdot \bm{H}^{(n)}(\bm{x}), \qquad 
	\bm{a}^{(n)} = \int f(\bm{x})\bm{H}^{(n)}(\bm{x}) d^nx
\end{equation}

Esta propiedad permite aplicar la expansi\'on en series de Hermite a la funci\'on de distribuci\'on de equilibrio en el espacio de velocidades
\begin{subequations}
	\begin{equation}
		f^{eq}(\rho, \bm{u}, \theta, \bm{\xi}) = \omega(\bm{\xi}) \sum_{n=0}^{\infty}\dfrac{1}{n!}\bm{a}^{(n)eq}(\rho, \bm{u}, \theta) \cdot \bm{H}^{(n)}(\bm{\xi})
	\end{equation}
	\begin{equation}
		\bm{a}^{(n)eq}(\rho, \bm{u}, \theta) = \int f^{eq}(\rho, \bm{u}, \theta, \bm{\xi})\bm{H}^{(n)}(\bm{\xi}) \mbox{d}^d \xi
	\end{equation}
	\label{eq:feq_hermite}
\end{subequations}

En particular, puede verse que la funci\'on de distribuci\'on de equilibrio tiene la misma forma funcional que la funci\'on generatriz $\omega(\bm{x})$
\begin{equation}
	f^{eq}(\rho, \bm{u}, \theta, \bm{\xi}) = \dfrac{\rho}{\theta^{d/2}} \, \omega \left( \dfrac{\bm{\xi} - \bm{u}}{\sqrt{\theta}} \right),
\end{equation}
de modo que el c\'alculo de los coeficientes $\bm{a}^{(n)}$ puede simplificarse mediante:
\begin{equation}
	\bm{a}^{(n)eq} = \rho \int \omega(\bm{\eta})\bm{H}^{(n)}(\sqrt{\theta}\bm{\eta} - \bm{u}) \mbox{d}^d \eta,
\end{equation}
donde $\bm{\eta} = (\bm{\xi} - \bm{u})/\sqrt{\theta}$. El c\'alculo de estas integrales puede realizarse directamente, de modo que los primeros coeficientes resultan:
\begin{subequations}
	\begin{align}
		a^{(0),eq}          &= \rho              \\ 
		a^{(1),eq}_{\alpha} &= \rho u_{\alpha}   \\
		a^{(2),eq}_{\alpha\beta} &= \rho \left[ u_{\alpha} u_{\beta} + (\theta-1)\delta_{\alpha\beta} \right]   \\
		a^{(3),eq}_{\alpha\beta\gamma} &= \rho \left[ u_{\alpha} u_{\beta} u_{\gamma} + (\theta-1)(\delta_{\alpha\beta}u_{\gamma} + \delta_{\beta\gamma}u_{\alpha} + \delta_{\gamma\alpha}u_{\beta}) \right].
	\end{align}
	\label{eq:eq_coeffs}
\end{subequations}

A partir de la \eq{eq:eq_coeffs} puede observarse que los coeficientes de la serie de Hermite est\'an directamente relacionados con las principales cantidades conservadas. En esta l\'inea, puede demostrarse que existe una relaci\'on similar para los coeficientes de la expansi\'on en series de Hermite de la funci\'on de distribuci\'on $f$:
\begin{subequations}
	\begin{align}
		a^{(0),eq} &= \int f^{eq} \mbox{d}^d \xi  &= \rho  &= \int f \mbox{d}^d \xi &= a^{(0)}   \\ 
		a^{(1),eq}_{\alpha} &= \int \xi_{\alpha}f^{eq} \mbox{d}^d \xi  &= \rho u_{\alpha}  &= \int \xi_{\alpha}f \mbox{d}^d \xi &= a^{(1)}_{\alpha}   \\ 		
		\dfrac{a^{(2),eq}_{\alpha\alpha}+\rho d}{2} &= \int \dfrac{|\bm{\xi}|^2}{2} f^{eq} \mbox{d}^d \xi  &= \rho E  &= \int \dfrac{|\bm{\xi}|^2}{2} f \mbox{d}^d \xi &= \dfrac{a^{(2)}_{\alpha\alpha}+\rho d}{2}
	\end{align}
\end{subequations}

La representaci\'on adecuada de las leyes de conservaci\'on macrosc\'opica puede alcanzarse con pocos t\'erminos de las series de Hermite, aunque se ha observado que la inclusi\'on de t\'erminos de mayor orden contribuyen a mejorar la precisi\'on y estabilidad del m\'etodo num\'erico final \cite{dhumieres_multiple-relaxation-time_2002}. De esta forma, la representaci\'on en serie de $f^{eq}$ con s\'olo $N=3$ t\'erminos puede aproximarse por:
\begin{align}
	f^{eq} &\approx \omega(\bm{\xi}) \sum_{n=0}^{N=3} \dfrac{1}{n!}\bm{a}^{(n),eq} \cdot \bm{H}^{(n)}(\bm{\xi}) \\
	&\approx \omega(\bm{\xi}) \rho \left[ 1 + \xi_{\alpha}u_{\alpha} + \left( u_{\alpha}u_{\beta}+(\theta-1)\delta_{\alpha\beta} \right)\left(\xi_{\alpha}\xi_{\beta} - \delta_{\alpha\beta}\right) \right]
\end{align}


\subsection{Discretizaci\'on de la funci\'on de distribuci\'on de equilibrio}

La expansi\'on de la funci\'on de distribuci\'on de equilibrio \feqvar{} en series de Hermite es apropiada, ya que la forma funcional de $f^{eq}(\bm{\xi})$ es similar a la de la funci\'on generatriz $\omega(\bm{xi})$, y los primeros coeficientes de la serie est\'an directamente relacionados con los principales momentos conservados (densidad, velocidad y energ\'ia). Por otro lado, el empleo de polinomios de Hermite permite calcular integrales de determinadas funciones utilizando la evaluaci\'on de dicha funci\'on en un intervalo discreto de puntos (abscisas), mediante la regla conocida como cuadratura de Gauss-Hermite. En particular, esta t\'ecnica permite calcular exactamente ciertas integrales de polinomios mediante:
\begin{equation}
	\int \omega(\bm{x}) P^{(N)}(\bm{x})\mbox{d}^dx = \sum_{i=1}^{n} w_i P^{(N)}(\bm{x}_i)
\end{equation}
donde $P^{(N)}$ es un polinomio de grado $N$, $n$ es al menos $n=(N+1)/2$, y $w_i$ son pesos asociados a las abscisas $\bm{x}_i$. En este caso, cada componente del punto multidimensional $\bm{x}_i$, es decir, $x_{i\alpha}$ con $\alpha=1\ldots d$, es una ra\'iz del polinomio de Hermite unidimensional $H^{n}(x_{i\alpha})=0$. De esta forma, la cuadratura de Gauss-Hermite puede usarse para reescribir los coeficientes de la serie de $f^{eq}$ mediante un conjunto discreto de velocidades $\{ \bm{\xi}_i \}$:
\begin{equation}
	\bm{a}^{(n),eq} = \int f^{eq}(\bm{\xi}) \bm{H}^{(n)}(\bm{\xi}) \mbox{d}^d \xi 
	= \rho \sum_{i=1}^n w_i Q(\bm{\xi}_i)\bm{H}^{(n)}(\bm{\xi}_i)
\end{equation}

Esta discretizaci\'on lleva a describir $n$ cantidades $f_i^{eq}(\bm{x},t)$, correspondientes a la funci\'on de distribuci\'on de equilibrio evaluada en la velocidad $\bm{\xi}_i$. Por lo tanto, podemos reemplazar a la funci\'on continua $f^{eq}(\bm{\xi})$ por un conjunto discreto
\begin{equation}
	f_i^{eq} = w_i \rho \left[ 1 + \xi_{i\alpha}u_{\alpha} + \dfrac{1}{2}\left( u_{\alpha}u_{\beta}+(\theta-1)\delta_{\alpha\beta} \right)\left(\xi_{i\alpha}\xi_{i\beta} - \delta_{\alpha\beta}\right) \right]
\end{equation}
\par
El conjunto ${f_i^{eq}}$ es continuo en espacio y tiempo, y satisface las mismas leyes de conservaci\'on para los primeros tres momentos de $f^{eq}(\bm{\xi})$. Finalmente, asumiendo un comportamiento isot\'ermico ($\theta = 1$) y reescribiendo las velocidades de las part\'iculas como
\begin{equation}
	\bm{e}_i = \dfrac{\bm{\xi}_i}{\sqrt{3}},
\end{equation}
podemos escribir una forma final para la distribuci\'on de equilibrio discreta:
\begin{equation}
	f_i^{eq} = w_i \rho \left[ 1 + \dfrac{e_{i\alpha}u_{\alpha}}{c_s^2} + \dfrac{u_{\alpha}u_{\beta}(e_{i\alpha}e_{i\beta}-c_s^2\delta_{\alpha\beta})}{2c_s^4} \right]
\end{equation}
donde adem\'as se defini\'o convenientemente a la constante $c_s$, llamada velocidad del sonido.


\subsection{Discretizaci\'on de la funci\'on de distribuci\'on}
El procedimiento aplicado para aproximar la dependencia de $f^{eq}$ en el espacio de velocidades $\bm{\xi}$ tambi\'en puede ser usado con la funci\'on de distribuci\'on $f$:
\begin{equation}
	\bm{a}^{(n)}(\bm{x},t) = \int f(\bm{x}, \bm{e}, t) \bm{H}^{(n)}(\bm{e}) \, \mbox{d}^d e \approx \sum_{i=1}^q f_i(\bm{x},t)\bm{H}^{(n)}(\bm{e}_i)
\end{equation}

Ahora se tiene un conjunto de $q$ funciones $f_i(\bm{x},t)$, relacionadas con las velocidades discretas $\bm{e}_i$ y continuas en el espacio y tiempo. Usando este conjunto es posible reescribir la ecuaci\'on de Boltzmann, pero esta vez discreta en el espacio de velocidades:
\begin{equation}
	\partial_t f_i + e_{i\alpha} \partial_{\alpha} f_{i} = \Omega(f_i), \qquad i=1\ldots q,
	\label{eq:boltz_disc_vel}
\end{equation}
donde los momentos macrosc\'opicos se pueden calcular usando sumas discretas:
\begin{subequations}
	\begin{equation}
		\rho        = \sum_i f_i          = \sum_i f_i^{eq}
	\end{equation}
	\begin{equation}
		\rho \bm{u} = \sum_i f_i\bm{e}_i  = \sum_i f_i^{eq}\bm{e}_i
	\end{equation}
\end{subequations}


\subsection{Conjunto discreto de velocidades}
La descomposici\'on de las funciones de distribuci\'on usando series de Hermite mostr\'o que el espacio de velocidades puede ser discretizado, pero hasta este punto no se estableci\'o de qu\'e manera. Los conjuntos de velocidades $\{\bm{e}_i\}$ admisibles deben cumplir dos propiedades fundamentales; por un lado, presentar una resoluci\'on suficiente que permita capturar los fen\'omenos f\'isicos deseados, y por otro contener la menor cantidad de componentes posibles para reducir el costo computacional involucrado. 
\par 
Tradicionalmente, los conjuntos de velocidades suelen identificarse con la notaci\'on D$d$Q$q$ introducida por \cite{qian_lattice_1992}, donde $d$ corresponde al n\'umero de dimensiones espaciales y $q$ a la cantidad de velocidades discretas. Estos conjuntos quedan determinados por las velocidades $\{\bm{e}_i\}$, los pesos $\{w_i\}$ y la velocidad del sonido $c_s^2$. Si bien existen numerosos mecanismos para construir conjuntos de velocidades con las propiedades deseadas, la alternativa m\'as sencilla y directa consiste en evaluar la isotrop\'ia rotacional de los tensores de grilla \cite{guo_lattice_2013,frisch_lattice_1987}, es decir, de aquellos momentos con factores de peso $\{w_i\}$. Esta simetr\'ia implica que los tensores de grilla de hasta orden 5 satisfagan
\begin{subequations}
	\begin{align}
	\sum_i w_i &= 1 \\
	\sum_i w_i e_{i\alpha} &= 0 \\
	\sum_i w_i e_{i\alpha} e_{i\beta} &= c_s^2 \delta_{\alpha\beta} \\
	\sum_i w_i e_{i\alpha} e_{i\beta} e_{i\gamma} &= 0 \\
	\sum_i w_i e_{i\alpha} e_{i\beta} e_{i\gamma} e_{i\mu} &= c_s^4(\delta_{\alpha\beta}\delta_{\gamma\mu} + \delta_{\alpha\gamma}\delta_{\beta\mu} + \delta_{\alpha\mu}\delta_{\beta\gamma}) \\
	\sum_i w_i e_{i\alpha} e_{i\beta} e_{i\gamma} e_{i\mu} e_{i\nu} &= 0.
	\end{align}
	\label{eq:tensores_grilla}	
\end{subequations}

Una vez establecidas este tipo de restricciones, el procedimiento habitual consiste en definir el conjunto de velocidades discretas, y posteriormente determinar $\{w_i\}$ y $c_s^2$. La dimensi\'on de $\{ \bm{e}_i\}$ depender\'a de la cantidad de restricciones de la \eq{eq:tensores_grilla} que quieran satisfacerse simult\'aneamente: para resolver adecuadamente ecuaciones macrosc\'opicas como Navier-Stokes se necesita cumplir con los primeros 6 tensores de grilla, mientras que para ecuaciones de advecci\'on-difusi\'on lineales, alcanza con satisfacer los primeros 4.
\par
La \eq{eq:boltz_disc_vel} suele discretizarse grillas espaciales regulares de espaciado $\Delta x$, y en intervalos de tiempo regulares $\Delta t$. Por lo tanto, es conveniente elegir el conjunto de velocidades $\{e\}_i$ de modo que conecten exclusivamente nodos vecinos. De esta forma surgen los modelos de grilla tradicionales como D1Q3, D2Q9 y D3Q15, los cuales se ilustran en la \textbf{figura de velocidades de grilla}. En la \tb{tab:DdQq} se resumen las principales propiedades de cada modelo de grilla.

\begin{table}[ht]
	\centering
    \begin{tabular}{c c c c}
	    \toprule
        \bf Modelo & $\{\bm{e}_i\}$ & $\{w_i\}$ & $c_s^2$ \\
        \midrule
        \multirow{2}{*}{D1Q3} & (0)      & 2/3 & \multirow{2}{*}{1/$\sqrt{3}$} \\
                              & ($\pm$1) & 1/6 &  \\                 
        \midrule
        \multirow{3}{*}{D2Q9} & (0,0) & 4/9 & \multirow{3}{*}{1/$\sqrt{3}$} \\
                              & ($\pm$1,0), (0,$\pm$1) & 1/9 &  \\
                              & ($\pm$1,$\pm$1) & 1/36 &  \\                              
        \midrule
        \multirow{3}{*}{D3Q15} & (0,0,0) & 2/9 & \multirow{3}{*}{1/$\sqrt{3}$} \\
                               & ($\pm$1,0,0), (0,$\pm$1,0), (0,0,$\pm$1) & 1/9 &  \\
                               & ($\pm$1,$\pm$1,$\pm$1) & 1/72 &  \\         
        \bottomrule
	\end{tabular}
	\caption{Ejemplos de conjuntos de velocidades}
	\label{tab:DdQq}
\end{table}  


\section{Discretizaci\'on del espacio y tiempo}  
Hasta este punto, s\'olo se aplic\'o la discretizaci\'on de la ecuaci\'on de Boltzmann en el espacio de velocidades. El paso final hacia la ELB debe completarse con la discretizaci\'on del espacio y tiempo.
\par 
La ecuaci\'on de Boltzman discreta (\eq{eq:boltz_disc_vel}) es una ecuaci\'on diferencial en derivadas parciales (EDP) de primer orden y parab\'olica. Una de las t\'ecnicas m\'as usadas en la resoluci\'on de este tipo de ecuaciones es aquella que se conoce como m\'etodo de las caracter\'isticas, que consiste en parametrizar las variables independientes de la EDP para transformarla en una ecuaci\'on diferencial ordinaria (EDO). En este caso, es posible expresar la soluci\'on de \ref{eq:boltz_disc_vel} como $f_i=f_i(\bm{x}(\zeta),t(\zeta))$, donde $\zeta$ parametriza una trayectoria en el espacio. De esta manera, la \eqref{eq:boltz_disc_vel} puede reescribirse usando un diferencial total:
\begin{equation}
	\dfrac{df}{d\zeta} = \dfrac{\partial f}{\partial t}\dfrac{\partial t}{\partial \zeta} + \dfrac{\partial f}{\partial x_{\alpha}}\dfrac{\partial x_{\alpha}}{\partial \zeta} = \Omega_i(\bm{x}(\zeta),t(\zeta))
	\label{eq:zeta_total}
\end{equation}

Por inspecci\'on, debe cumplirse
\begin{equation}
	\dfrac{\partial t}{\partial \zeta} = 1, \qquad
	\dfrac{\partial x_{\alpha}}{\partial \zeta} = e_{i\alpha}.
\end{equation}
de modo que las soluciones $f_i$ siguen una trayectoria dada por $\bm{x}=\bm{x}_0 + \bm{e_i}t$, donde $\bm{x}_0$ es una constante arbitraria. Si se considera la trayectoria que pasa a trav\'es del punto $(\bm{x}_0,t_0)$, con $t(\zeta=0)=t_0$ y $\bm{x}(\zeta=0)=\bm{x}_0$, entonces la integraci\'on de la \eq{eq:zeta_total} resulta:
\begin{equation}
	f_i(\bm{x}_0+\bm{e_i}\Delta t,t_0+\Delta t)-f_i(\bm{x}_0,t_0)=\int_0^{\Delta t}\Omega_i(\bm{x}_0+\bm{e_i}\zeta,t_0+\zeta) \, \mbox{d}\zeta.
\end{equation}

Como el punto $(\bm{x}_0,t_0)$ es arbitrario, la integraci\'on puede generalizarse como:
\begin{equation}
	f_i(\bm{x}+\bm{e_i}\Delta t,t+\Delta t)-f_i(\bm{x},t)=\int_0^{\Delta t}\Omega_i(\bm{x}+\bm{e_i}\zeta,t+\zeta) \, \mbox{d}\zeta.
	\label{eq:elb_integral}
\end{equation}

A partir de la integraci\'on de la \eq{eq:elb_integral} resulta explicito el acople entre la discretizaci\'on espacial y temporal, y refuerza la practicidad de emplear conjuntos de velocidades que, en un intervalo de tiempo $\Delta t$, se vinculen con las posiciones vecinas en la grilla espacial.
\par 
S\'olo resta integral el t\'ermino derecho de la \eq{eq:elb_integral}. Empleando em m\'etodo de Euler expl\'icito, puede obtenerse finalmente la ecuaci\'on de lattice Boltzmann
\begin{equation}
	f_i(\bm{x}+\bm{e_i}\Delta t,t+\Delta t)-f_i(\bm{x},t)=\Delta t\Omega_i(\bm{x},t).
	\label{eq:elb}
\end{equation}

La discretizaci\'on de Euler empleada conduce a una aproximaci\'on de primer orden en la discretizaci\'on de espacio y tiempo. Sin embargo, puede demostrarse que si se realiza dicha integraci\'on mediante el m\'etodo trapezoidal \cite{he_discrete_1998}, y con una redefinici\'on adecuada de la funci\'on de distribuci\'on discreta, es posible obtener una ecuaci\'on igual a \ref{eq:elb}. Por lo tanto, es posible afirmar que la \eq{eq:elb} constituye una aproximaci\'on de segundo orden tambi\'en en espacio y tiempo.

\section{Operadores de colisi\'on}
LBGK y MRT. Matrices de transformaci\'on?

\section{La expansi\'on de Chapman-Enskog}
Podemos ponerla ac\'a, aunque hay que ver como queda con las cuentas m\'as adelante.
Hay que ver, pero podr\'ian ir ac\'a las cuentas de la ecuaci\'on b\'asica.

\section{Overview de LBM}
Algoritmo, colision, streaming, etc.

\chapter{Simulaci\'on de flujo multif\'asico}

\section{Lattice Boltzmann para flujo multif\'asico}
En sinton\'ia con el crecimiento de la mec\'anica de fluidos computacional, fueron desarroll\'andose numerosos m\'etodos num\'ericos macrosc\'opicos destinados a resolver las ecuaciones de Navier-Stokes en flujos multif\'asicos \cite{scardovelli_direct_1999}. Entre los m\'etodos m\'as populares, pueden destacarse el de \textbf{front-tracking}, el m\'etodo Volume of Fluid (VOF) y el m\'etodo level set. A pesar de la amplia difusi\'on adquirida, y de la demostrada capacidad para resolver con precisi\'on diversos escenarios con flujos multif\'asicos, estas t\'ecnicas tradicionales contin\'uan presentando limitaciones que dificultan el modelado de problemas complejos con transferencia de calor, como ebullici\'on y condensaci\'on. En particular, el m\'etodo de \textbf{front-tracking} generalmente no permite simular adecuadamente procesos de coalescencia y ruptura de una interfase \cite{scardovelli_direct_1999,liu_three-dimensional_2012}. La aplicaci\'on de VOF y level set suele requerir pasos de reconstrucci\'on o reinicializaci\'on de la interfase, que pueden no ser f\'isicos y complejos de implementar \cite{liu_three-dimensional_2012}. Adem\'as, suelen originarse inestabilidades num\'ericas en el uso de VOF o level set para simular flujos dominados por tensi\'on superficial en geometr\'ias complejas \cite{scardovelli_direct_1999}.
\par
En comparaci\'on con otros m\'etodos computacionales, el MLB presenta ventajas adicionales para la simulaci\'on de flujos complejos. Por un lado, la naturaleza mesosc\'opica con base en la teor\'ia cin\'etica molecular permite generar modelos con s\'olidos fundamentos termodin\'amicos. Por otro lado, es posible incorporar directamente el uso de ecuaciones de estado en la resoluci\'on de Navier-Stokes en escala macrosc\'opica, lo que a su vez elimina la necesidad de resolver una ecuaci\'on de Poisson para la presi\'on. Finalmente, la mayor\'ia de los modelos son sencillos de programar, y la naturaleza local de las operaciones involucradas facilita la explotaci\'on de arquitecturas con paralelismo masivo, como las unidades de procesamiento gr\'afico (GPU).
\par 
Las mencionadas caracter\'isticas han motivado el desarrollo de esquemas para flujo multif\'asico desde los or\'igenes del m\'etodo. A pesar de que se ha conformado un enorme universo de modelos diferentes, la gran mayor\'ia de estas alternativas pueden agruparse dentro de cuatro categor\'ias principales: color-gradient \cite{liu_three-dimensional_2012,gunstensen_lattice_1991}, pseudopotential \cite{shan_lattice_1993,shan_simulation_1994,chen_critical_2014}, free-energy \cite{swift_lattice_1996,inamuro_galilean_2000} y phase-field \cite{he_lattice_1999,liang_phase-field-based_2014}. 

\subsection*{Color-gradient}
El m\'etodo color-gradient fue introducido por Gunstensen et al. \cite{gunstensen_lattice_1991}, como una versi\'on mejorada del modelo LGA multif\'asico de Rothman y Keller \cite{rothman_immiscible_1988}. En este modelo las fases se denotan con diferentes colores, y la interacci\'on entre part\'iculas, responsable de la separaci\'on de fases, es modelada con gradientes locales de color asociado a la diferencia de densidad entre ambas fases. Tomando como ejemplo un sistema de dos fases, el modelo color-gradient original usa dos tipos de funciones de distribuci\'on, $f_{ri}$ y $f_{bi}$, para representar a los fluidos rojo y azul respectivamente. La distribuci\'on total de la mezcla $f_i = f_{ri}+f_{bi}$ evoluciona como:
\begin{equation}
	f_i(\bm{x}+\bm{e}_i\delta_t,t+\delta_t) - f_i(\bm{x},t) = \Omega_i^c + \Omega_i^p,
\end{equation}
donde $\Omega_i^c$ denota los efectos de colisi\'on y $\Omega_i^p$ se encuentra relacionado con la tensi\'on interfacial. En este caso, las densidades y velocidades para cada fase se definen como
\begin{equation}
	\begin{gathered}
	\rho_k = \sum_i f_{ki}, \qquad \rho_k\bm{u}_k = \sum_i \bm{e}_if_{ki}, \qquad k=r,b, \\
	\rho = \rho_r + \rho_b, \qquad \rho\bm{u} = \rho_r\bm{u}_r + \rho_b\bm{u}_b.
	\end{gathered}
\end{equation}

Si bien es posible emplear un operador LBGK para $\Omega_i^c$, el t\'ermino $\Omega_i^p$ se calcula empleando un par\'ametro de orden que tiene en cuenta la diferencia de densidad entre fases. Despu\'es de la colisi\'on, las funciones de distribuci\'on parciales son sometidas a un paso de ajuste de color antes del streaming. Estos pasos adicionales del algoritmo contribuyen a producir inestabilidades num\'ericas, a la vez que reducen de forma dr\'astica la eficiencia computacional por paso de tiempo \cite{guo_lattice_2013}.

\subsection*{Free-energy}
El m\'etodo free-energy fue propuesto originalmente por Swift et al. \cite{swift_lattice_1996}, y presenta un punto de partida asociado a consideraciones termodin\'amicas b\'asicas. La idea detr\'as de estos m\'etodos consiste en derivar una funci\'on de distribuci\'on de equilibrio adecuada, de forma que el momento de segundo orden correspondiente incluya un tensor de presi\'on termodin\'amico no ideal. En particular, este tensor se deriva a partir de la energ\'ia libre de un fluido asociado a una ecuaci\'on de estado de Van der Waals, y puede escribirse como:
\begin{equation}
	P^{'}_{\alpha\beta}=p\delta_{\alpha\beta}+\kappa\dfrac{\partial \rho}{\partial x_{\alpha}}\dfrac{\partial \rho}{\partial x_{\beta}}
\end{equation}
donde $\kappa$ es una constante asociada al valor de tensi\'on superficial en la interfase. Para lograr la recuperaci\'on de este tensor, Swift et al. sugirieron el uso de una ELB con operador de colisi\'on LBGK
\begin{equation}
	f_i(\bm{x}+\bm{e}_i\delta_t,t+\delta_t) - f_i(\bm{x},t) = -\dfrac{1}{\tau}\left[ f_i - f_i^{eq}(\rho, \bm{u}, \nabla\bm{u}) \right],
\end{equation}
mientras que la distribuci\'on de equilibrio satisface las siguientes restricciones:
\begin{equation}
	\sum_i f_i^{eq} = \rho, \qquad \sum_i \bm{e}_if_{i}=\rho\bm{u}, \qquad
	\sum_i \bm{e}_i\bm{e}_if_{i}= \bm{P}^{'}+\rho\bm{u}\bm{u}
	\label{eq:free_energy+rest}
\end{equation}

De forma similar a lo que ocurre con el modelo est\'andar para flujos de una \'unica fase, $f^{eq}$ puede escribirse como un polinomio de $\bm{u}$:
\begin{subequations}
	\begin{equation}
		f_i^{eq}=A + B(\bm{e}_i \cdot \bm{u}) + C u^2 + D(\bm{e}_i \cdot \bm{u})^2 + G:\bm{e}_i\bm{e}_i, \qquad i\neq 0,
	\end{equation}
	\begin{equation}
		f_i^{eq}=A_0 + C_0 u^2, \qquad i = 0.
	\end{equation}
\end{subequations}

Los coeficientes de $f_i^eq$ para una grilla bidimensional pueden obtenerse empleando las restricciones de la \eq{eq:free_energy+rest}:
\begin{equation}
	\begin{gathered}
		A_0 = \rho - 6 A, \qquad C_0 = -\rho, \\
		A = \dfrac{1}{3}(p_0-\kappa \rho \nabla^2 \rho), \qquad B = \dfrac{\rho}{3}, \qquad C = -\dfrac{\rho}{6}, \qquad  D = \dfrac{2\rho}{3},  \\
		G_{xx} = -G_{yy} = \dfrac{\kappa}{3}\left[ \left(\dfrac{\partial \rho}{\partial x}\right)^2 - \left(\dfrac{\partial \rho}{\partial y}\right)^2 \right], \qquad G_{xy}=\dfrac{2\kappa}{3}\dfrac{\partial \rho}{\partial x} \dfrac{\partial \rho}{\partial y}.
	\end{gathered}
\end{equation}

De esta forma, las ecuaciones macrosc\'opicas recuperadas usando el m\'etodo de Swift resultan:
\begin{equation}
	\dfrac{\partial \rho}{\partial t} + \nabla \cdot (\rho \bm{u}) = 0,
\end{equation}
\begin{align}
	\dfrac{\partial \rho \bm{u}}{\partial t} + \nabla \cdot (\rho \bm{uu}) =& -\nabla p_0 + \nu \nabla^2 (\rho \bm{u})+\nabla[\lambda \nabla \cdot (\rho \bm{u})]		\\
	&-\left( \tau - \dfrac{1}{2} \right)\dfrac{\partial p_0}{\partial \rho} \delta_t \nabla \cdot [\bm{u}\nabla \rho + (\nabla \rho)\bm{u}],
\end{align}
donde 
\begin{equation}
	\nu = \dfrac{\delta_t}{4}\left( \tau - \dfrac{1}{2} \right), \qquad \lambda = \delta_t\left( \tau - \dfrac{1}{2} \right)\left( \dfrac{1}{2} - \dfrac{\partial p_0}{\partial \rho}\right)
\end{equation}

\textcolor{red}{Falta decir q\'ue es $p_0$.}

Las primeras versiones asociadas a esta familia sufrieron la falta de invariancia galileana debido a la recuperaci\'on de t\'erminos que no est\'an relacionados con Navier-Stokes, originados por la misma incorporaci\'on del tensor de presi\'on en la distribuci\'on de equilibrio \cite{kuzmin_multi-relaxation_2008}. Para recuperar esta invarianza es necesario, por lo tanto, incorporar t\'erminos de correcci\'on en la funci\'on de distribuci\'on de equilibrio. Este tipo de adaptaciones son similares a aquellas adoptadas por las versiones posteriores de los m\'etodos dentro de la familia color-gradient, y usualmente constituyen fuentes adicionales de inestabilidad num\'erica al incorporar t\'erminos como $\bm{u}\nabla\rho$ y $\bm{u}\cdot\nabla\rho$.

\subsection*{Phase-field}
Esta categor\'ia representa a los modelos basados en la teor\'ia de phase-field, es decir, aquellos en los que la din\'amica de la interfase se encuentra descripta por un par\'ametro de orden regido por una ecuaci\'on de Cahn-Hilliard o similar \cite{jacqmin_calculation_1999}. Esta aproximaci\'on a la simulaci\'on de flujos multif\'asicos con LB tiene su contraparte equivalente dentro de las t\'ecnicas tradicionales de CFD para modelos de interfase difusa, como el de Ding et. al \cite{ding_diffuse_2007}.

La versi\'on original de He at al. \cite{he_lattice_1999} hace uso de dos funciones de distribuci\'on, $g$ y $f$, para recuperar las ecuaciones de Navier-Stokes y una del tipo Cahn-Hilliard para la evoluci\'on de la interfase respectivamente. Usando operadores de colisi\'on LBGK, las ecuaciones corresponden a:
\begin{equation}
	g_i(\bm{x}+\bm{e}_i\delta_t,t+\delta_t) - g_i(\bm{x},t) = -\dfrac{1}{\tau_1}\left[ g_i(\bm{x},t) - g_i^{eq}(\bm{x},t) \right] + S_i(\bm{x},t)\delta_t,
	\label{eq:he_g_eq}
\end{equation}
\begin{equation}
	f_i(\bm{x}+\bm{e}_i\delta_t,t+\delta_t) - f_i(\bm{x},t) = -\dfrac{1}{\tau_2}\left[ f_i(\bm{x},t) - f_i^{eq}(\bm{x},t) \right] + S_i^{'}(\bm{x},t)\delta_t,
\end{equation}
donde la viscosidad cinem\'atica se recupera mediante $\nu=(c_s^2)(\tau_1-0.5)\delta_t$, $\tau_2$ se relaciona con la mobilidad de la ecuaci\'on de Cahn-Hilliard, y $S_i$, $S_i^{'}$ son t\'erminos de fuente. Las funciones de distribuci\'on de equilibrio se definen mediante:
\begin{equation}
	g_i^{eq} = w_i\left[ p + \rho c_s^2 \left( \dfrac{e_{i\alpha}u_{\alpha}}{c_s^2}  + \dfrac{e_{i\alpha}u_{\alpha}e_{i\beta}u_{\beta}}{2c_s^4} - \dfrac{u_{\alpha}u_{\alpha}}{2c_s^2} \right)  \right]
\end{equation}
\begin{equation}
	f_i^{eq} = w_i \phi \left[ 1 +  \dfrac{e_{i\alpha}u_{\alpha}}{c_s^2}  + \dfrac{e_{i\alpha}u_{\alpha}e_{i\beta}u_{\beta}}{2c_s^4} - \dfrac{u_{\alpha}u_{\alpha}}{2c_s^2}  \right]
\end{equation}
donde $p$ corresponde a la presi\'on hidrodin\'amica. $\phi$ es el par\'ametro de orden que se utiliza, por ejemplo, para determinar la distribuci\'on de densidad:
\begin{equation}
	\rho(\phi) = \rho_g + \dfrac{\phi - \phi_g}{\phi_l - \phi_g}(\rho_l - \rho_g)
\end{equation}

En este caso, los sub\'indices $l$ y $g$ corresponden a las fases l\'iquida y gaseosa respectivamente. Las variables macrosc\'opicas del modelo de He et al. se calculan como:
\begin{equation}
	\begin{gathered}
		\phi = \sum_i f_i \\
		p = \sum_i g_i - \dfrac{\delta_t}{2} u_{\beta} \dfrac{\partial(p - \rho c_s^2)}{\partial x_{\beta}} \\
		\rho u_{\alpha}c_s^2 = \sum_i e_{i\alpha}g_i + \dfrac{\delta_t}{2}c_s^2 F_{\alpha}
	\end{gathered}
	\label{eq:he_macro_variables}
\end{equation}
donde $F_{\alpha}$ representa las fuerzas externas, incluyendo las asociadas a la tensi\'on interfacial. La expansi\'on de Chapman-Enskog de las Ecs.~\eqref{eq:he_g_eq}-\eqref{eq:he_macro_variables} muestra que las ecuaciones macrosc\'opicas recuperadas resultan:
\begin{equation}
	\begin{gathered}
		\dfrac{\partial (\rho \bm{u})}{\partial t} + \nabla \cdot (\rho \bm{uu})  = -\nabla p  + \nu \nabla \cdot \left[ \rho (\nabla\bm{u} + \nabla \bm{u}^T) \right] + \bm{F} \\
		\dfrac{\partial \phi}{\partial t} + \nabla \cdot (\phi \bm{}u) = \dfrac{1}{2} \left( 1 - \dfrac{1}{2\tau_2} \right) \nabla^2 (p - c_s^2 \phi)
	\end{gathered}
\end{equation}

\subsection*{Pseudopotential}
El m\'etodo pseudopotencial, que podr\'ia considerarse como la t\'ecnica m\'as sencilla para simular flujos multif\'asicos, fue propuesta por Shan y Chen \cite{shan_lattice_1993,shan_simulation_1994}. En este m\'etodo, las interacciones entre part\'iculas fluidas son imitadas mediante un potencial interpart\'icula, de modo que la separaci\'on de fases ocurre autom\'aticamente, sin mecesidad de recurrir a t\'ecnicas para capturar o reconstruir interfases. Este potencial es el responsable de inducir un tensor de presi\'on no ideal, diferente al del m\'etodo free-energy. La simplicidad conceptual y la elevada eficiencia computacional convirtieron a este m\'etodo en uno de los m\'as pupulares, habiendo sido utilizado con \'exito en la resoluci\'on de diversos problemas.
\par 
La ELB propuesta por Shan y Chen conserva la estructura est\'andar de los modelos de \'unica fase con operador LBGK:
\begin{equation}
	f_i(\bm{x}+\bm{e}_i\delta_t,t+\delta_t) - f_i(\bm{x},t)= -\dfrac{1}{\tau}\left[ f_i(\bm{x},t) - f_i^{eq}(\rho,\bm{u}^{eq}) \right],
\end{equation}
donde $\bm{u}^{eq}$ se conoce como velocidad de equilibrio. En este modelo, los principales momentos quedan definidos por:
\begin{equation}
	\sum_i f_i = \rho, \qquad	\sum_i \bm{e}_if_i = \rho \bm{u}^* .
\end{equation}

En este caso, la velocidad $\bm{u}^*$ es utilizada para calcular $\bm{u}^{eq}$ y la velocidad real del fluido
\begin{equation}
	\bm{u}^{eq} = \bm{u}^* + \dfrac{\bm{F}\tau}{\rho}, \qquad \bm{u} = \bm{u}^* + \dfrac{\bm{F}\delta_t}{2\rho}
\end{equation}

El modelo original de Shan y Chen introduce el uso de una fuerza de interacci\'on entre part\'iculas vecinas definida como:
\begin{equation}
	\bm{F}_{int}(\bm{x},t) = -G\psi(\bm{x},t)\sum_i w_i \psi(\bm{x}+\bm{e}_i,t)\bm{e}_i
\end{equation}
donde $G$ es un par\'ametro que controla la intensidad de la fuerza de interacci\'on y $\psi$ es un potencial dado por:
\begin{equation}
	\psi(\rho) = \rho_0 \left[ 1-\mbox{e}^{-\frac{\rho}{\rho_0}} \right],
\end{equation}
donde $\rho_0$ es una constante arbitraria.

A pesar de la evoluci\'on y mejora de los diversos modelos multif\'asicos desde sus or\'igenes, siguen existiendo diferencias significativas en las capacidades de simulaci\'on de problemas multif\'asicos din\'amicos, sobre todo cuando las relaciones de densidad entre fases ($\rho_l/\rho_g$) son elevadas. Como se menciona en el trabajo de Li et al. \cite{li_lattice_2016}, estas limitaciones pueden deberse a diferentes causas. Por un lado, existen diferencias en las cantidades f\'isicas que deben ser evualuadas a trav\'es de la interfase l\'iquido-vapor; por ejemplo densidad en el modelo free-energy y potencial en pseudopotential. Por otro lado, c\'omo se mencion\'o previamente, las familias color-gradient y free-energy necesitan correcciones adicionales para recuperar adecuadamente el comportamiento de Navier-Stokes, y como estos t\'erminos dependen expl\'icitamente del gradiente de densidad, suelen convertirse en fuentes de inestablidad num\'erica \cite{leclaire_unsteady_2014,leclaire_enhanced_2013,huang_simulations_2013}. Este hecho origina que los modelos color-gradient y free-energy encuentren severas limitaciones al momento de simular flujos con elevada relaci\'on de densidades y alto n\'umero de Reynolds, a pesar de los \'exitos observados en casos est\'aticos o cuasi-est\'aticos.

A diferencia de color-gradient y free-energy, los modelos multif\'asicos agrupados dentro de las familias phase-field y pseudopotential han sido aplicados exitosamente en la simulaci\'on de problemas con elevada relaci\'on de densidades y n\'umero de Reynolds modederados, como impacto y colisi\'on de droplets \cite{li_lattice_2013,lee_stable_2005} e incluso aplicaciones sencillas de transferencia de calor con cambio de fase \cite{safari_consistent_2014,markus_pool_2012,gong_lattice_2015}. En este \'ultimo aspecto es d\'onde la familia de modelos pseudopotential presenta su mayor virtud: como se describe en las secciones siguientes, el potencial de interacci\'on puede modificarse para incorporar ecuaciones de estado arbitrarias, de modo que los procesos asociados a la transferencia de masa entre fases quedan determinados exclusivamente por  dicha ecuaci\'on. Adem\'as, a diferencia de los modelos phase-field, si se elije una ELB adecuada para recuperar macrosc\'opicamente una ecuaci\'on de energ\'ia, entonces no es necesario reconstruir la interfase para estimar la fuente de masa en la ecuaci\'on de impulso correspondiente \cite{safari_consistent_2014,safari_extended_2013}. De esta forma, los modelos LB para flujos multif\'asicos basados en la familia pseudopotential permiten conservar la simplicidad y eficiencia computacional representativa de \'este m\'etodo, a\'un en la simulaci\'on de flujos complejos.


\section{El modelo pseudopotential}
Originalmente, Shan y Chen introdujeron una interacci\'on no local entre part\'iculas fluidas, definiendo a la fuerza experieanentada por las part\'iculas en la posici\'on $\bm{x}$ respecto aquellas en $\bm{x}'$ como:
\begin{equation}
	\bm{F}(\bm{x},\bm{x}') = -\tilde{G}(|\bm{x}-\bm{x}'|)\psi(\bm{x})\psi(\bm{x'})(\bm{x}-\bm{x}'),
	\label{eq:fint_green}
\end{equation}
donde $\tilde{G}$ es una funci\'on de Green y $\psi$ una masa efectiva que depende de la densidad local. La estructura de la fuerza dada por la \eq{eq:fint_green} fue dise\~nada adecuadamente por Shan y Chen, ya que si bien este acoplamiento entre masas efectivas no conserva el impulso local durante el proceso de colisi\'on, puede demostrarse que conserva el impulso total y, por lo tanto, no introduce impulso neto al sistema \cite{shan_simulation_1994}. La fuerza de interacci\'on total actuando sobre las part\'iculas en $\bm{x}$ resulta:
\begin{equation}
	\bm{F}(\bm{x}) = -\psi(\bm{x}) \sum \tilde{G}(|\bm{x}-\bm{x}'|)\psi(\bm{x}')(\bm{x}-\bm{x}')
\end{equation}

En un espacio discreto puede considerarse que cada nodo interact\'ua con $N$ vecinos, y si se asume que esta interacci\'on es isotr\'opicam es decir $\tilde{G} = \tilde{G}(|\bm{e}_{\alpha}|)$, entonces puede expresarse a la fuerza de interacci\'on discreta como:
\begin{equation}
	\bm{F}_i = -G\psi(\bm{x})c_s^2 \sum_{\alpha=1}^N \omega(|\bm{e}_{\alpha}|^2)\psi(\bm{x}+\bm{e}_{\alpha}\delta_t)\bm{e}_{\alpha},
	\label{eq:f_int}
\end{equation}
donde $G$  es la magnitud de la interacci\'on y $\{\omega(|\bm{e}_{\alpha}|^2)\}$ son coeficientes asociados a la discretizaci\'on isotr\'opica de $\bm{F}$. Estos pesos son diferentes de 	aquellos empleados en la funci\'on de distribuci\'on de equilibrio est\'andar (\eq{eq:feq}).

La expansi\'on en serie de Taylor de la \eq{eq:f_int} muestra que los t\'erminos dominantes est\'an dados por \cite{shan_pressure_2008}:
\begin{equation}
	\bm{F}_i=-G\left[ c_s^2 \psi \nabla \psi + \dfrac{c_s^4}{2} \psi \nabla (\nabla^2 \psi)  + \ldots \right]
	\label{eq:f_int_taylor}
\end{equation}

Por lo tanto, a pesar de su aparente simplicidad, esta fuerza de interacci\'on es capaz de incorporar elementos de fluidos no ideales, como una ecuaci\'on de estado relacionada con el primer t\'ermino del miembro derecho de la \eq{eq:f_int_taylor}, y el efecto de tensi\'on superficial relacionado con el segundo t\'ermino.

\begin{itemize}
	\item Ecuaci\'on recuperada
	Origen del nombre pseudopotential
\end{itemize}


\subsection{Densidades de coexistencia y la regla de construcci\'on de Maxwell}

\textcolor{red}{En la secci\'on anterior} se mostr\'o que los modelos pseudopotential permiten recuperar, a nivel macrosc\'opico, una ecuaci\'on de conservaci\'on de impulso lineal en donde se identifica un tensor de presi\'on que depende de la masa reducida o potencial de interacci\'on. De esta manera, la elecci\'on de un potencial de interacci\'on determina directamente la dependencia de la presi\'on de equilibrio con propiedades macrosc\'opicas del fluido, como densidad y temperatura, constituyendo una ecuaci\'on de estado para el fluido simulado. 

La expresi\'on para este potencial no es \'unico, y es evidente que su elecci\'on est\'a ligada al comportamiento de cada fase. Por lo tanto, antes de evaluar el efecto de este potencial en el resultado de la aplicaci\'on de un modelo pseudopotencial, es necesario remarcar cu\'ales son los v\'inculos entre flujos multif\'asicos y ecuaciones de estado termodin\'amicas.

El an\'alisis de un sistema multif\'asico, como agua l\'iquida y vapor de agua, busca responder aspectos termodin\'amicos fundamentales sobre la condici\'on de equilibrio l\'iquido-vapor. En particular, es necesario establecer una relaci\'on entre las densidades de la fase l\'iquida ($\rho_l$) y de la fase gaseosa o de vapor ($\rho_g$), as\'i como la dependencia de la presi\'on con estas densidades. Este requerimiento f\'isico de mantener una coexistencia de fases impone una restricci\'on adicional en la ecuaci\'on de estado, es decir, en la ley que describe la compleja interdependencia entre la presi\'on $p$, los vol\'umenes molares $v$ (o densidades, ya que $v \propto 1/\rho$), y la temperatura $T$. 

Las ecuaciones de estado m\'as conocidas tienen su origen en la teor\'ia cin\'etica de gases \cite{blundell_concepts_2006}. El modelo de gases reales m\'as utilzado es el de van der Waals (vdW), el cual resulta a su vez el m\'as simple pero que permite incorporar dos ingredientes cruciales: interacciones moleculares y mol\'eculas de tama\~no distinto de cero. La ecuaci\'on de estado de van der Waals es:
\begin{equation}
	\left( p + \dfrac{a}{v^2} \right) \left( V_m-b \right) = RT,
\end{equation}
donde $p$ es la presi\'on termodin\'amica, $T$ la temperatura, $R$ la constante universal de gases, y $v$ el volumen molar, es decir, el volumen ocupado por 1 mol de mol\'eculas del gas. En esta ecuaci\'on, la constante $a$ parametriza la intensidad de interacci\'on entre mol\'eculas, mientras que la constante $b$ incorpora el efecto del tama\~no finito de las mol\'eculas. Si $a$ y $b$ son nulos, la ecuaci\'on de vdW se reduce a la de gases ideales, es decir, $pv = RT$. Adem\'as, este comportamiento tambi\'en se recupera en el l\'imite de densidades muy bajas ($v \gg b$ y $v \gg \sqrt{a/\rho}$). Por otro lado, cuando la densidad es alta y $v$ se aproxima a $b$, el valor de presi\'on $p$ diverge.

La \red{Figura} ejemplifica el comportamiento de la ecuaci\'on de van der Waals, donde cada l\'inea corresponde a una isoterma. A medida que la temperatura desciende, las isotermas pasan de un comportamiento similar a un gas ideal, en la esquina superior derecha de la figura, a exhibir un comportamiento en forma de S con un m\'inimo y un m\'aximo locales, en la esquina inferior izquierda. De acuerdo a esas isotermas, existen tres valores de densidad asociadas a una misma presi\'on. Sin embargo, en esa zona de S existe una regi\'on donde $(\partial v / \partial \rho)_T$ es positiva, y por lo tanto una compresibilidad negativa, lo que implica que esa regi\'on es inestable frente a perturbaciones de presi\'on \cite{burden_numerical_2011}.
La temperatura a partir de la cual se produce este cambio de comportamiento se conoce como temperatura cr\'tica, y es la que se ilustra con una l\'inea m\'as gruesa en la \red{Figura}. Esta isoterma presenta un punto de inflexi\'on, conocido como punto cr\'itico, sobre el que se definen propiedades caracter\'isticas de la ecuaci\'on de estado. En particular, analizando las derivadas $(\partial p / \partial v)_T = 0$ y $(\partial^2 p / \partial v^2)_T = 0$, puede encontrarse cu\'al es ese punto cr\'itico, es decir, los valores de presi\'on cr\'itica, volumen molar cr\'itico y temperatura cr\'itica que lo caracterizan:
\begin{equation}
	\begin{gathered}
		p_c = \dfrac{a}{27 b^2} \\
		v_c = 3b \\
		T_c = \dfrac{8 a}{27 R b}
	\end{gathered}
	\label{eq:vdw_param_crit}
\end{equation}

En la \red{Figura} se evidencia que para temperaturas por debajo del valor cr\'itico, dos vol\'umenes molares diferentes pueden adoptar el mismo valor de presi\'on en equilibrio $p_0$ (en realidad son 3, pero uno es inestable). Por lo tanto, puede establecerse un estado de coexistencia entre dos fases, l\'iquido y vapor, que compartan igual energ\'ia libre de Gibbs (o equivalentemente potencial qu\'imico). Esta restriccici\'on implica:
\begin{equation}
	\int_{v_l}^{v_g} \left[p_0 - p(v',T)\right] \, \mbox{d} v' = 0,
	\label{eq:maxwell_constr}
\end{equation}
donde $p_0 = p(v_l,T) = p(v_g,T)$. La \eq{eq:maxwell_constr} se conoce como regla de construcci\'on de Maxwell, y determina los vol\'umenes de coexistencia de ambas fases para una determinada temperatura. Si se realiza un cambio de variables adecuado, la \eq{eq:maxwell_constr} puede reescribirse como:
\begin{equation}
	\int_{\rho_g}^{\rho_l} \left[p_0 - p(\rho',T)\right] \dfrac{1}{\rho'} \, \mbox{d} \rho' = 0,
	\label{eq:maxwell_constr_rho}
\end{equation}
donde $p_0 = p(\rho_l,T) = p(\rho_g,T)$. Gr\'aficamente, resolver la regla de Maxwell implica encontrar los vol\'umenes para los que las \'areas sombreadas de la \red{Otra figura} sean iguales. 

La regla de construcci\'on de Maxwell aplica a cualquier ecuaci\'on de estado, e impone una restricci\'on termodin\'amica a las simulaciones realizadas con modelos pseudopotential: si se desea recuperar el comportamiento de un fluido regido por una ecuaci\'on de estado, entonces la ``separaci\'on autom\'atica de fases'' propia de esta familia debe ser capaz de reproducir las densidades de equilibrio dadas por la \eq{eq:maxwell_constr}.


\subsubsection*{Otras ecuaciones de estado}



\subsubsection*{Ley universal}






\subsection{Incorporaci\'on de ecuaciones de estado}

La definici\'on de masa efectiva dada por Shan y Chen se aproxima a un valor asint\'otico cuando la densidad es alta, lo que contribuye a evitar el colapso de la fase de mayor densidad y, por lo tanto, contribuir a mejorar la estabilidad de las simulaciones. Sin embargo, el uso de esta expresi\'on fija la ecuaci\'on de estado, lo que limita la aplicabilidad a diferentes fluidos. En 2006. Yuan y Schaefer \cite{yuan_equations_2006} demostraron que es posible alcanzar relaciones de densidad elevadas en casos est\'aticos si se emplea la masa efectiva introducida por He y Doolen \cite{he_thermodynamic_2002}:
\begin{equation}
	\psi(\rho) = \sqrt{\dfrac{2(p_{EOS} - \rho c_s^2)}{Gc^2}},
\end{equation}
donde $p_{EOS}$ representa una ecuaci\'on de estado no ideal, como Van der Waals, Carnahan-Starling o Peng-Robinson.






\subsection{La condici\'on de estabilidad mec\'anica y el problema de inconsistencia termodn\'amica}

En el caso sin fuerzas de interacci\'on, a nivel macrosc\'opico se recupera una ecuaci\'on de estado ideal de la forma $p=\rho c_s^2$. Sin embargo, la incorporaci\'on de \ref{eq:f_int} produce una nueva ecuaci\'on de estado dependiente del potencial de interacci\'on:
\begin{equation}
	p = \rho c_s^2 + \dfrac{G}{2} c_s^2 \psi ^2,
\end{equation}
mientras que el tensor de presi\'on $\bm{P}$ queda definido como \cite{he_thermodynamic_2002}:
\begin{equation}
	\nabla \cdot \bm{P} = \nabla \cdot (\rho c_s^2 \bm{I}) - \bm{F}.
	\label{eq:ptens}
\end{equation}

Shan \cite{shan_pressure_2008} demostr\'o que para garantizar un balance mec\'anico exacto, debe considerarse una forma discreta del tensor de presi\'on. En particular, esta expresi\'on puede derivarse a partir de una integral de volumen de la \eq{eq:ptens}:
\begin{equation}
	\int (\nabla \cdot \bm{P}) \, \mbox{d}\Omega = \int \nabla \cdot (\rho c_s^2 \bm{I})\, \mbox{d}\Omega - \int \bm{F} \, \mbox{d}\Omega,
	\label{eq:integral_pres}
\end{equation}
donde $\Omega$ es un volumen cerrado. Si se aplica el teorema de integraci\'on de Gauss a la \eq{eq:integral_pres} resulta
\begin{equation}
	\int \bm{P} \, \mbox{d} \bm{A} = \int \rho c_s^2 \bm{I}\, \mbox{d} \bm{A} - \int \bm{F} \, \mbox{d}\Omega,
\end{equation}
donde $\mbox{d} \bm{A}$ es un diferencial de \'area. En forma discreta, esta integral puede escribirse como
\begin{equation}
	\sum \bm{P} \cdot \bm{A} = \sum \rho c_s^2 \bm{I} \cdot \bm{A} - \sum \bm{F}.
\end{equation}

Por lo tanto, el tensor de presi\'on finalmente puede escribirse como:
\begin{equation}
	\bm{P} = \rho c_s^2 \bm{I} + \dfrac{G}{2}\psi(\bm{x}) \sum_{\alpha=1}^N w(|\bm{e}_{\alpha}|^2)\psi(\bm{x}+\bm{e}_{\alpha})\bm{e}_{\alpha}\bm{e}_{\alpha}.
	\label{eq:ptens_shan}
\end{equation}

En el caso de considerar s\'olo interacciones con los vecinos cercanos (ubicados a una distancia m\'axima de $\sqrt{2}$ o $\sqrt{3}$ unidades de grilla en 2 o 3 dimensiones respectivamente), la expansi\'on en serie de Taylor de la \eq{eq:ptens_shan} resulta:
\begin{equation}
	\bm{P} = \left( \rho c_s^2 + \dfrac{G c^2}{2} \psi^2 + \dfrac{G c^4}{12} \psi \nabla^2 \psi \right) \bm{I} + \dfrac{G c^4}{6} \psi \nabla \nabla \psi.
	\label{eq:ptens_shan_taylor}	
\end{equation}

La \eq{eq:ptens_shan_taylor} puede usarse para determinar la presi\'on normal a una interfase plana
\begin{equation}
	P_n = \rho c_s^2 + \dfrac{G c^2}{2} \psi^2 + \dfrac{G c^4}{12} \left[ \alpha \left( \dfrac{d\psi}{dn} \right)^2 + \beta \psi \dfrac{d^2 \psi}{dn^2} \right],
	\label{eq:ptens_shan_plane}	
\end{equation}
donde $n$ denota la direcci\'on normal a la interfase, y en el caso de interacci\'on de vecinos cercanos se cumple $\alpha = 0$ y $\beta = 3$. Por lo tanto, tomando como base la \eq{eq:ptens_shan_plane} y considerando que en equilibrio la presi\'on $P_n$ debe ser igual a la presi\'on est\'atica en el seno del fluido \cite{shan_pressure_2008}, puede derivarse la condici\'on de estabilidad mec\'anica:
\begin{equation}
	\int_{\rho_g}^{\rho_l} \left( p_0 - \rho c_s^2 - \dfrac{Gc^2}{2} \psi^2 \right) \dfrac{\psi'}{\psi^{1+\varepsilon}} \, \mbox{d}\rho = 0,
\end{equation}
donde $\psi' = d\psi / d\rho$, $\varepsilon=-2\alpha/\beta$ y $p_0=p(\rho_l)=p(\rho_g)$. De esta forma, puede verse que la condici\'on de estabilidad mec\'anica conduce a densidades de coexistencia que en general ser\'an diferentes a las establecidas por la construcci\'on de Maxwell (\textcolor{red}{poner numero de ecuaci\'on previa}), ya que si el potencial de interacci\'on satisface la propuesta de He y Doolen, entonces en general no puede cumplirse que $\psi' / \psi^{1+\varepsilon} \propto 1/\rho^2$. Esta incompatibilidad se conoce con el nombre de inconsistencia termodin\'amica.






%% % Podemos usar cualquiera de los dos comandos: \input o \include para incluir el texto
%% \input{cap1}
%% \include{cap2}


%% \appendix
%% \include{apend1}

 \begin{biblio}
    \bibliographystyle{ibtesis}
 	\bibliography{Library}
 \end{biblio}


%% \begin{postliminary}

%% \begin{seccion}{Publicaciones asociadas}
%%   \begin{enumerate}
%%   \item Mi primer aviso en la revista \textbf{ABC}, 1996
%%   \item Mi segunda publicaci\'{o}n en la revista \textbf{ABC}, 1997
%%   \end{enumerate}
%% \end{seccion}

%% \begin{seccion}{Agradecimientos}
%% A todos los que se lo merecen, por merecerlo
%% \end{seccion}

%% \end{postliminary}

\end{document}

