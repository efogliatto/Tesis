\chapter{Conclusiones generales}

El m\'etodo de lattice Boltzmann ha evolucionado hasta convertirse en una herramienta robusta y fiable para la simulaci\'on num\'erica de la mec\'anica de fluidos. Desde un punto de vista formal, LB puede entenderse como un camino para resolver una ecuaci\'on de Boltzmann: en lugar de discretizar directamente aquellas ecuaciones de inter\'es, el m\'etodo propone transportar una funci\'on de distribuci\'on en una grilla regular, considerando que los momentos (en el espacio de fases) de esta \fdp{} est\'an asociados a las principales magnitudes hidrodin\'amicas de las ecuaciones macrosc\'opicas de inter\'es, como densidad, velocidad o temperatura. A trav\'es de t\'ecnicas de expansi\'on multiescala, como el an\'alisis de Chapman-Enskog, es posible verificar que una adecuada elecci\'on de grilla espacial, velocidades discretas del espacio de fases, y operador de colisi\'on, pueden ser combinadas adecuadamente para obtener la soluci\'on de ecuaciones de conservaci\'on t\'ipicas de la mec\'anica de fluidos. 

El camino de la t\'ecnica converge en un algoritmo con dos caracter\'isticas distintivas que constituyen su principal atractivo: simplicidad de implementaci\'on y eficiencia computacional. Nuevos modelos de lattice Boltzmann pueden construirse en torno a premisas b\'asicas, como la preservaci\'on del esquema de advecci\'on y colisi\'on num\'erica de la \fdp{}. De esta manera, el m\'etodo puede expandirse potencialmente hacia la resoluci\'on de fen\'omenos complejos, como el transporte de energ\'ia en flujos multif\'asicos, a la vez que se aprovecha la escalabilidad de modernos sistemas de c\'omputo.

\bigskip

La simulaci\'on de flujos multif\'asicos con transferencia de calor involucra generalmente una resoluci\'on por etapas. Por un lado es necesario resolver una ecuaci\'on de lattice Boltzmann para representar el transporte de las variables macrosc\'opicas hidrodin\'amicas, como densidad y velocidad. La transferencia de energ\'ia, por otro lado, suele abordarse con una estrategia diferente.

En este cap\'itulo se analiz\'o la primera de estas etapas, tomando como base al conjunto de modelos de la familia pseudopotencial. En este m\'etodo se introduce una fuerza de interacci\'on entre part\'iculas de fluido vecinas, que depende de un potencial de interacci\'on asociado a ecuaciones de estado, como la de van der Waals. Esta fuerza origina la separaci\'on entre fases, sin necesidad de recurrir a t\'ecnicas adicionales para capturar o reconstruir la interfase.

Considerando las ventajas que presenta la familia pseudopotencial, se opt\'o por analizar y evaluar el modelo MRT bidimensional de Li y colaboradores mediante la simulaci\'on de problemas que presentan soluci\'on anal\'itica. En primer lugar, se verific\'o la capacidad del modelo de reproducir la curva de coexistencia de fases de la ecuaci\'on de estado de van der Waals. A partir de la simulaci\'on de dominios peri\'odicos sin fuerzas externas, se encontr\'o que el modelo es capaz de reproducir las densidades de coexistencia para un amplio rango de temperaturas, mediante el ajuste adecuado del par\'ametro libre $\sigma$. Estos resultados no se encuentran influenciados significativamente por el valor de los factores de relajaci\'on, aunque $\sigma$ debe ser adaptado a cada conjunto de constantes de la ecuaci\'on de estado.

Por otro lado, se introdujo una expresi\'on expl\'icita para evaluar la distribuci\'on de densidad en un fluido van der Waals dentro de una cavidad unidimensional, bajo acci\'on de la gravedad. Los resultados de las simulaciones mostraron que el modelo de Li reproduce adecuadamente los perfiles de densidad en un amplio rango de condiciones de simulaci\'on, como magnitud del campo gravitatorio o diferentes distribuciones de temperatura en el interior de la cavidad.

La soluci\'on del problema de estratificaci\'on de un fluido van der Waals es unidimensional, pero introduce un mecanismo rico y robusto para evaluar la precisi\'on de modelos pseudopotenciales. La comparaci\'on con una expresi\'on anal\'itica permite discriminar aspectos espec\'ificos de la simulaci\'on, como los efectos de las constantes de la ecuaci\'on de estado en la resoluci\'on de la interfase. En este aspecto, se verific\'o que para una misma temperatura y gravedad, el espesor de la interfase depende de estas constantes y no cambia con la cantidad de nodos de la grilla. Por lo tanto, se observ\'o que los cambios en la representaci\'on de la interfase debido a modificaciones en la cantidad de elementos de grilla, s\'olo puede cuantificarse de manera consistente mediante la representaci\'on de la soluci\'on num\'erica en unidades reducidas. Este resultado, observado en detalle en la simulaci\'on de un problema multif\'asico sin transferencia de calor, constituye un elemento fundamental para el an\'alisis de lattice Boltzmann como m\'etodo num\'erico, especialmente en la simulaci\'on de fen\'omenos de mayor complejidad.

\bigskip

En este cap\'itulo se introdujo un nuevo esquema de lattice Boltzmann en dos dimensiones, con dos funciones de distribuci\'on y operador de colisi\'on MRT, destinado a la resoluci\'on de flujos multif\'asicos con transferencia de calor. Este modelo acopla una \lbe{} de la familia \pp{} para resolver las ecuaciones hidrodin\'amicas, junto con una ecuaci\'on modificada para el transporte de energ\'ia.

A diferencia de los esquemas de LB tradicionales, la estrategia propuesta introduce una adecuada definici\'on de la distribuci\'on de equilibrio directamente en el espacio de momentos, con par\'ametros libres que pueden usarse para ajustar la difusividad t\'ermica recuperada. Adem\'as, el an\'alisis de Chapman-Enskog muestra que cuando esta distribuci\'on de equilibrio se emplea junto con una matriz de relajaci\'on con coeficientes extra-diagonales no nulos, adecuadamente definidos, la ecuaci\'on de energ\'ia macrosc\'opica se recupera sin t\'erminos adicionales hasta la escala de expansi\'on analizada. De esta forma, no es necesario aplicar correcciones expl\'icitas a los t\'erminos de fuente o a la distribuci\'on post-colisi\'on, usualmente usadas para eliminar los efectos relacionados con la simulaci\'on de ecuaciones escalares de advecci\'on-difusi\'on usando esquemas LB cl\'asicos.

El modelo propuesto es capaz de reproducir problemas num\'ericos con soluciones anal\'iticas, como distribuci\'on de temperatura y densidad en un fluido vdW estratificado, as\'i como la evoluci\'on de la interfase en un problema de Stefan unidimensional. En el primer caso, el modelo es capaz de calcular adecuadamente las distribuciones en el centro del fluido y la posici\'on de la interfase para diferentes condiciones de contorno. De forma similar a lo observado en el caso isot\'ermico, se observa consistencia del m\'etodo en unidades reducidas. En la segunda prueba, la evoluci\'on de la posici\'on de la interfase pudo ser reproducida satisfactoriamente usando diferentes combinaciones de constantes de equilibrio y par\'ametros de relajaci\'on que conducen a una misma difusividad t\'ermica, lo que evidencia el cumplimiento del comportamiento previsto por la expansi\'on de Chapman-Enskog.

Por otro lado, se analiz\'o la consistencia y orden de convergencia del modelo propuesto mediante la simulaci\'on de generaci\'on de burbujas sobre una superficie horizontal calefaccionada. Para ello, se aplic\'o una evaluaci\'on de independencia de grilla de dos pasos, tomando como base la reproducci\'on de n\'umeros adimensionales caracter\'isticos. En primer lugar, se realizaron simulaciones sobre diferentes grillas, conservando estos par\'ametros junto con el espesor de interfase adimensional. En este caso, no se observaron diferencias significativas en la evoluci\'on de la interfase entre las distintas simulaciones, lo que constituye un claro indicio de que las ecuaciones macrosc\'opicas se recuperan seg\'un lo esperado. Adicionalmente, se realiz\'o un segundo conjunto de simulaciones sobre diferentes grillas, conservando los mismos n\'umeros adimensionales relevantes pero conservando el espesor de interfase en unidades de grilla. En este caso, el di\'ametro de partida adimensional converge con orden 2.18.

El problema de ebullici\'on heterog\'enea sirvi\'o adem\'as que la elecci\'on del m\'etodo usado para aplicar las condiciones de contorno influye significativamente en el proceso de crecimiento y desprendimiento de las burbujas. En particular, mientras que el modelo de Inamuro produce resultados consistentes con simulaciones efectuadas con una ecuaci\'on de energ\'ia resuelta por diferencias finitas, los m\'etodos de equilibrio y de extrapolaci\'on no conducen al rompimiento del cuello, y por lo tanto, evitan el desprendimiento de las burbujas. 

Durante el desarrollo del modelo t\'ermico se propuso una alternativa para calcular el gradiente de temperatura expl\'icitamente en el t\'ermino de fuente, que s\'olo involucra a los valores locales de la funci\'on de distribuci\'on y de la distribuci\'on de equilibrio en cada nodo. Con este esquema, no se observaron diferencias significativas con los resultados obtenidos usando diferencias finitas centradas para $\nabla T$, lo que motiva al uso de este esquema en caso de requerir una optimizaci\'on de modelos de c\'alculo paralelizados.

Finalmente, simulaciones de ebullici\'on sobre placas planas, donde se considera una condici\'on de contorno de temperatura con fluctuaciones en torno a un valor medio, muestran que el modelo permite reproducir, a nivel cuantitativo, fen\'omenos caracter\'isticos del proceso. Si bien estas simulaciones corresponden a dominios bidimensionales, y por lo tanto sin nexos directos con un caso real, se observ\'o que modificaciones en la temperatura de la superficie calefactora induce la generaci\'on de burbujas siguiendo patrones significativamente diferentes. En particular, en las condiciones de simulaci\'on presentadas se observ\'o una producci\'on de burbujas asociadas a los reg\'imenes de ebullici\'on nucleada y de ebulluci\'on de pel\'icula. 


\bigskip

La utilizaci\'on de ecuaciones con operador de colisi\'on MRT en la simulaci\'on del transporte de energ\'ia en flujos multif\'asicos, provee una alternativa flexible para el desarrollo de modelos t\'ermicos vinculados a los modelos isot\'ermicos de la familia \pp{}. En particular, la propuesta de \lbe{} construida sobre una conjunto de velocidades D2Q9 pudo ser extendida a grillas D3Q15, considerando una adaptaci\'on adecuada de la distribuci\'on de equilibrio, t\'ermino de fuente y elementos extra-diagonales de la matriz de relajaci\'on. De esta manera, se obtuvo una \lbe{} que permite recuperar la ecuaci\'on de energ\'ia de M\'arkus y H\'azi sin t\'erminos no deseados, con una difusividad t\'ermica recuperada que puede ajustarse con los factores de relajaci\'on o, alternativamente, con par\'ametros libres asociados a la distribuci\'on de equilibrio.

De manera similar a lo observado para la versi\'on bidimensional, el nuevo modelo es capaz de reproducir la distribuci\'on de densidad y temperatura para un fluido con \eos{} de vdW dentro de una cavidad bajo acci\'on de la cavidad. Los resultados obtenidos con este modelo, nuevamente, muestran consistencia al ser analizados empleando unidades reducidas. Por otro lado, se observ\'o que esta propuesta permite simular adecuadamente el problema de Stefan unidimensional, demostrando que es posible recuperar adecuadamente el calor latente y la difusividad t\'ermica asociados al uso arbitrario de las constantes de la \eos{} y de los par\'ametros libres de la \edf{}.