\chapter{Transferencia de calor en flujo multif\'asico bidimensional}

En general, los modelos existentes de lattice Boltzmann para simular transferencia de calor en flujos multif\'asicos pueden clasificarse en tres categor\'ias principales: modelos de velocidad m\'ultiple (\emph{multi-speed}), modelos h\'ibridos (\emph{hybrid}) y modelos con dos funciones de distribuci\'on (\emph{double-distribution}). En el primer caso, las propiedades macrosc\'opicas del fluido como densidad, velocidad y temperatura, se obtienen a partir de los momentos cero, primero y segundo de una \'unica funci\'on de distribuci\'on \cite{alexander_lattice_1993,succi_lattice_2018}. A pesar que estos modelos cuentan con un s\'olido fundamento te\'orico, su uso difusi\'on es limitada, ya que t\'ipicamente emplean un set de velocidades mayor que el correspondiente modelo isot\'ermico. suelen incorporar t\'erminos de mayor orden en la funci\'on de distribuci\'on de equilibrio, y est\'an restringidos a la simulaci\'on de un fluido con n\'umero de Prandtl fijo.

Estas limitaciones intr\'insecas de los modelos de velocidad m\'ultiple motivaron el desarrollo de otro tipo de modelos t\'ermicos, donde se introduce un acoplamiento expl\'icito entre un esquema isot\'ermico para las ecuaciones hidrodin\'amicas, y una t\'ecnica alternativa para el transporte de energ\'ia. Los modelos h\'ibridos \cite{dong_numerical_2012,li_lattice_2015}, por un lado, aprovechan la discretizaci\'on del dominio asociada a la ecuaci\'on hidrodin\'amica, y hacen uso de un esquema c\'asico de diferencias finitas para encontrar la soluci\'on de la ecuaci\'on de energ\'ia correspondiente. Por otro lado, los modelos con dos funciones de distribuci\'on introducen una segunda \lbe{} para recuperar la ecuaci\'on macrosc\'opica deseada.

Los esquemas de doble funci\'on de distribuci\'on heredan las principales ventajas de lattice Boltzmann, como simplicidad del algoritmo y elevada eficiencia computacional. Sin embargo, tambi\'en presentan las limitaciones comunes derivadas de la representaci\'on de ecuaciones escalares de advecci\'on-difusi\'on \cite{markus_simulation_2011, huang_modified_2014, li_improved_2017, huang_numerical_2011, li_effect_2014, huang_multiphase_2015}. En este tipo de ecuaciones hay una \'unica cantidad conservada (por ejemplo temperatura), de modo que se relajan las condiciones de isotrop\'ia y es posible, en teor\'ia, emplear distribuciones de equilibrio lineales (en la velocidad del fluido) o conjuntos de velocidades reducidos. Sin embargo, la expansi\'on de Chapman-Enskog de estos esquemas muestra que existen t\'erminos no deseados que se recuperan inevitablemente en las ecuaciones macr\'oscipicas \cite{huang_modified_2014, huang_numerical_2011, kruger_lattice_2017}, y en el caso de los modelos para flujo multif\'asico con transferencia de calor, estos t\'erminos producen fuentes de calor adicionales que dependen del potencial de interacci\'on o de derivadas temporales de la velocidad del fluido.

Un ejemplo t\'ipico de estas restricciones puede analizarse con el uso de una \lbe{} con operador de colisi\'on SRT destinada a recuperar una ecuaci\'on de advecci\'on difusi\'on de un escalar pasivo $\phi$: \red{D2Q9?}
\begin{equation}
	\dfrac{\partial \phi}{\partial t} + \nabla \cdot (\phi \bm{u}) = \nabla \cdot (k \nabla \phi),
	\label{eq:adv_dif_phi}
\end{equation}
donde $k$ es una constante de difusi\'on. En este caso, si se utiliza un esquema cl\'asico de lattice Boltzmann, es decir:
\begin{equation}
	\begin{gathered}
		g_{\alpha}(\bm{x}+\bm{e}_{\alpha}\delta_t, t+\delta_t) - g_{\alpha}(\bm{x},t) = -\dfrac{1}{\tau} \left[ g_{\alpha}(\bm{x},t) - g^{eq}_{\alpha}(\bm{x},t) \right], \\[2mm]
		g_{\alpha}^{eq} = w_i \phi \left[ 1 + \dfrac{e_{i\alpha}u_{\alpha}}{c_s^2} + \dfrac{u_{\alpha}u_{\beta}(e_{i\alpha}e_{i\beta}-c_s^2\delta_{\alpha\beta})}{2c_s^4} \right], \\[2mm]
		\phi = \sum_{\alpha} g_{\alpha},
	\end{gathered}
\end{equation}
entonces puede demostrarse que la ecuaci\'on macrosc\'opica recuperada para $\phi$ satisface:
\begin{equation}
	\dfrac{\partial \phi}{\partial t} + \nabla \cdot (\phi \bm{u}) = \nabla \cdot \left\{ \delta_t(\tau - 0.5) \left[ \dfrac{\partial (\phi\bm{u})}{\partial t} + \nabla \cdot (\phi \bm{uu}) + c_s^2 \nabla \phi\right]\right\}.
	\label{eq:adv_dif_phi_rec}
\end{equation}

En este caso, el coeficiente de difusi\'on recuperado est\'a dado por $k=c_s^2 \delta_t(\tau-0.5)$. A diferencia de la \eq{eq:adv_dif_phi}, la \eq{eq:adv_dif_phi_rec} contiene un t\'ermino de desviaci\'on dado por $\nabla \cdot \{ \delta_t(\tau - 0.5) [ \partial_t (\phi\bm{u}) + \nabla \cdot (\phi \bm{uu}) ]\}$, el cu\'al puede anularse completamente s\'olo en aquellos casos con perfiles de velocidad especiales, como $\bm{u}=const$, $\bm{u}=[u_x(y),0]$, $\bm{u}=[0,u_y(x)]$, etc. 

Ejemplos similares pueden encontrarse dentro de los modelos \pp{} \cite{li_effect_2014}. En ciertas aplicaciones se desea recuperar una ecuaci\'on macrosc\'opica de la forma:
\begin{equation}
	\dfrac{\partial (\rho c_v T)}{\partial t} + \nabla \cdot (\rho c_v T \bm{u}) = \nabla \cdot (\lambda \nabla T),
\end{equation}
donde $\lambda$ es la conductividad t\'ermica y $c_v$ el calor espec\'ifico a volumen constante. En este caso el esquema usual consiste en utilizar una \lbe{} \pp{} est\'andar para la funci\'on de distribuci\'on $f$, asociada a las ecuaciones hidrodin\'amicas, mientras que este esquema tambi\'en puede utilizarse para una segunda distribuci\'on $g$, con $g^{eq}_{\alpha} = c_v T f^{eq}_{\alpha}$ y $\sum g_{\alpha} = \rho c_v T$. Nuevamente, la expansi\'on de Chapman-Enskog demuestra que la ecuaci\'on macrosc\'opica recuperada resulta:
\begin{equation}
	\dfrac{\partial (\rho c_v T)}{\partial t} + \nabla \cdot (\rho c_v T \bm{u}) = \nabla \cdot \left(\lambda \nabla T + \dfrac{\lambda}{p}T\bm{F} \right),
	\label{eq:e_macro_F}
\end{equation}
es decir, aparece un t\'ermino no deseado asociado a la fuerza total de la ecuaci\'on de impulso. Si estos t\'erminos son identificados, como en este caso, entonces pueden ser compensados expl\'icitamente:
\begin{equation}
	g_{\alpha}(\bm{x}+\bm{e}_{\alpha}\delta_t, t+\delta_t) - g_{\alpha}(\bm{x},t) = -\dfrac{1}{\tau} \left[ g_{\alpha}(\bm{x},t) - g^{eq}_{\alpha}(\bm{x},t) \right] + \delta_t C_{\alpha}(\bm{x},t),
\end{equation}
donde $C_{\alpha}$ es un t\'ermino de correcci\'on:
\begin{equation}
	C_{\alpha} = \left( 1-\dfrac{1}{2\tau} \right)\omega_{\alpha}c_v T \dfrac{\bm{e}_{\alpha} \cdot \bm{F}}{c_s^2}.
\end{equation}

En definitiva, los modelos de doble funci\'on de distribuci\'on resultan atractivos, ya que permiten conservar la elegancia y simpleza propia de lattice Boltzmann, mientras que aportan una mayor flexibilidad en el tipo de ecuaciones macrosc\'opicas que pueden recuperarse. Sin embargo, si se desea incrementar la aplicabilidad de estos modelos hacia problemas m\'as complejos, es necesario desarrollar un mecanismo natural para eliminar los t\'erminos no deseados en la ecuaci\'on macrosc\'opica recuperada. Por lo tanto, siguiendo este objetivo primario, en la secci\'on siguiente se introduce un modelo alternativo que permite remediar satisfactoriamente estas limitaciones.


\section{Ecuaci\'on de energ\'ia con operador MRT}
El modelo \pp{} de Li ha sido extensamente validado en la bibliograf\'ia, donde se ha demostrado que puede utilizarse exitosamente en la simulaci\'on de flujos multif\'asicos bidimensionales, con $Re$ moderados y relaci\'on de densidades elevadas (hasta $\rho_l / \rho_g \sim 750$). Adicionalmente, los resultados de la simulaci\'on de un fluido van der Waals mostrados en el \chap{chap:isot} y en \cite{fogliatto_simulation_2019}, sugieren que el modelo es capaz de incorporar adecuadamente fuerzas externas sin afectar el efecto sobre la inconsistencia termodin\'amica. Adem\'as, el an\'alisis de convergencia en malla, que en este contexto se traduce en un incremento de unidades de grilla conservando par\'ametros adimensionales, muestra que es posible alcanzar una precisi\'on adecuada en la reproducci\'on de interfases. Sin embargo, si se desea expandir su uso hacia la simulaci\'on de flujos multif\'asicos con transferencia de calor y cambio de fase, entonces es necesario proveer un nuevo esquema para cuantificar adecuadamente el transporte de energ\'ia.

Dentro del contexto de modelos \pp{}, donde se consideran fluidos cuya presi\'on termodin\'amica se encuentra gobernada por una ecuaci\'on de estado, la ecuaci\'on de energ\'ia macrosc\'opica debe seguir una forma particular. Tomando como base un balance local de entrop\'ia, y despreciando efectos de disipaci\'on viscosa, M\'arkus y H\'azi derivaron una ecuaci\'on macrosc\'opica para la energ\'ia, representada por medio de la temperatura \cite{markus_simulation_2011}:
\begin{equation}
	\dfrac{\partial T}{\partial t} + \bm{u} \cdot \nabla T = \dfrac{1}{\rho c_v} \nabla \cdot(\lambda \nabla T) - \dfrac{T}{\rho c_v} \left( \dfrac{\partial p_{EOS}}{\partial T} \right)_{\rho} \nabla \cdot \bm{u},
	\label{eq:markus_orig}
\end{equation}
donde $\lambda$ es la conductividad t\'ermica y $c_v$ el calor espec\'ifico a volumen constante. La \eq{eq:markus_orig} consiste en una ecuaci\'on de advecci\'on-difusi\'on para $T$, con un t\'ermino de fuente asociado al trabajo de compresi\'on. En las secciones siguientes se considerar\'a que la difusividad t\'ermica $\chi = \lambda/(\rho c_v)$ y el calor espec\'ifico $c_v$ son constantes en cada fase, de modo que la \eq{eq:markus_orig} puede reescribirse como:
\begin{equation}
	\dfrac{\partial T}{\partial t} + \nabla \cdot (\bm{u} T) = \chi \nabla^2 T  + \dfrac{\chi}{\rho} \nabla T \cdot \nabla \rho + T \left[ 1 - \dfrac{1}{\rho c_v} \left( \dfrac{\partial p_{EOS}}{\partial T} \right)_{\rho} \right] \nabla \cdot \bm{u}
	\label{eq:markus}
\end{equation}

Esta aproximaci\'on, si bien implica una p\'erdida de generalidad, facilita el desarrollo del esquema de lattice Boltzmann. La forma original de la ecuaci\'on puede recuperarse usando t\'erminos de fuente adecuados, como se describe en las secciones siguientes. Para satisfacer esta restricci\'on es posible partir del an\'alisis sobre la simulaci\'on de ecuaciones de advecci\'on-difus\'on llevado a cabo por Huang y colaboradores \cite{huang_modified_2014, huang_numerical_2011, huang_lattice_2015}: la recuperaci\'on de una ecuaci\'on escalar usando una \lbe{} con operador de colisi\'on MRT permite combinar aquellos momentos responsables de los t\'erminos no deseados. Si esta interacci\'on se lleva a cabo de forma inteligente, entonces es posible lograr la supresi\'on de componentes espec\'ificos dentro de las escalas de expansi\'on analizadas. Por otro lado, como se desea recuperar un t\'ermino de fuente, \'este debe incorporarse al esquema sin que se reproduzcan derivadas convectivas de dicha fuente \cite{cheng_introducing_2008}, lo que fuerza a la incorporaci\'on de un t\'ermino de fuente impl\'icito. Combinando estas ideas, una \lbe{} adecuada para recuperar la \eq{eq:markus} resulta:

\begin{equation}
	\begin{aligned}
		\tilde{\bm{g}}(\bm{x} + \bm{e} \delta_t, t+\delta_t) &=
		\tilde{\bm{g}}(\bm{x}, t) - (\bm{M}^{-1}Q\bm{M}) \left[ \tilde{\bm{g}}(\bm{x}, t) - \tilde{\bm{g}}^{eq}(\bm{x}, t) \right] \\
		&  + \dfrac{\delta_t}{2} \hat{\bm{\Gamma}}(\bm{x},t) + \hat{\bm{\Gamma}}(\bm{x} +\bm{e}\delta_t,t + \delta_t),
	\end{aligned}
	\label{eq:g_tilde_2d}	
\end{equation}

\begin{equation}
	\sum_{\alpha} \tilde{g}_{\alpha} = T.
\end{equation}

Siguiendo la notaci\'on usada previamente para las \lbe{} con operador MRT, la componente $\alpha$-\'esima del miembro izquierdo de la \eq{eq:g_tilde_2d} corresponde  a $\tilde{g}_{\alpha}(\bm{x}+\bm{e}_{\alpha}\delta_t)$, y $\bm{M}$ a la matriz de transformaci\'on al espacio de momentos.  En este caso, $\bm{Q}$ corresponde a una matriz de coeficientes relajaci\'on y $\hat{\bm{\Gamma}} = \bm{M}^{-1}\bm{\Gamma}$ a un t\'ermino de fuente en el espacio de poblaciones. Como la \eq{eq:g_tilde_2d} contiene una fuente \'implicita, puede transformarse usando:
\begin{equation}
	g_{\alpha}(\bm{x},t) = \tilde{g}_{\alpha} (\bm{x},t) - \dfrac{\delta_t}{2} \hat{\Gamma}_{\alpha}(\bm{x},t),
	\label{eq:g_tilde_tranf_2d}
\end{equation}
de modo que el momento de orden cero de $\bm{g}$ resulta:
\begin{equation}
	\begin{aligned}
		T &= \sum_{\alpha} \tilde{g}_{\alpha} \\
		  &= \sum_{\alpha} g_{\alpha} + \dfrac{\delta_t}{2} \sum_{\alpha} \tilde{\Gamma}_{\alpha} \\
		  &= \sum_{\alpha} g_{\alpha} + \dfrac{\delta_t}{2} \Gamma_0
	\end{aligned}
	\label{eq:T_macro_2d}
\end{equation}

En la \eq{eq:T_macro_2d} se utiliza que la primera fila de $\bm{M}$ corresponde a $\bm{M}_0 = (1 \cdots 1)$, de forma que $T$ puede obtenerse usando la distribuci\'on $\bm{g}$ y la primera componente de la fuente en el espacio de momentos. La \eq{eq:g_tilde_2d} puede reescribirse usando la \eq{eq:g_tilde_tranf_2d}:
\begin{equation}
	\bm{g}(\bm{x} + \bm{e} \delta_t, t+\delta_t) = \bm{M}^{-1} \left[ \bm{n} - \bm{Q}(\bm{n} - \feq{n}) + \delta_t \left( \bm{I}-\dfrac{\bm{Q}}{2}\right) \bm{\Gamma}\right],
	\label{eq:g_model_2d}
\end{equation}
donde $\bm{n} = \bm{Mg}$ y se omiti\'o la dependencia $(\bm{x},t)$ por claridad. La determinaci\'on de la cuaci\'on macrosc\'opica  recuperada puede hacerse mediante una expansi\'on de Chapman-Enskog, similar a la mostrada en el \red{Cap2}. En primer lugar, es necesario reescribir el t\'ermino izquierdo de la \eq{eq:g_model_2d} empleando un desarrollo de Taylor para cada componente, es decir:
\begin{equation}
	g_{\alpha}(\bm{x}+\bm{e}_{\alpha}\delta_t,t+\delta_t) = \sum_{k=0}^{\infty} \dfrac{1}{k!}\delta_t^k D_{\alpha}^{k} \, g_{\alpha} \sim g_{\alpha} + \delta_t D_{\alpha} g_{\alpha} + \dfrac{1}{2}\delta_t^2 D_{\alpha}^2g_{\alpha},
	\label{eq:taylor_gral}
\end{equation}
donde $D_{\alpha} = \partial_t + \bm{e}_{\alpha} \cdot \nabla=\partial_t + e_{\alpha \beta} \partial_{\beta}$. Para escribir la \eq{eq:taylor_gral} en forma vectorial, es necesario definir las matrices diagonales $\bm{E}_{\beta} = diag(e_{0\beta} \cdots e_{q-1\beta})$ y $\bm{\nabla}_{\beta} = \partial_{\beta} \bm{I}$, de modo que la \eq{eq:taylor_gral} puede transformarse como:
\begin{equation}
	\bm{g}(\bm{x}+\bm{e}\delta_t. t+\delta_t) = \bm{g} 
	+ \delta_t\left( \dfrac{\partial}{\partial t} \bm{I} + \bm{E}_{\beta}\bm{\nabla}_{\beta} \right) \bm{g}
	+ \dfrac{\delta^2_t}{2} \left( \dfrac{\partial}{\partial t} \bm{I} + \bm{E}_{\beta}\bm{\nabla}_{\beta} \right)^2 \bm{g}.
	\label{eq:taylor_vec}
\end{equation}

Reemplazando la \eq{eq:taylor_vec} en la \eq{eq:g_model_2d} y multiplicando ambos miembros por $\bm{M}$, se obteniene una ecuaci\'on desarrollada en torno a $(\bm{x},t)$:
\begin{equation}
	\begin{aligned}
	\delta_t \bm{M}\left( \dfrac{\partial}{\partial t} \bm{I} + \bm{E}_{\beta}\bm{\nabla}_{\beta} \right) \bm{g} + \dfrac{\delta^2_t}{2} \bm{M}\left( \dfrac{\partial}{\partial t} \bm{I} + \bm{E}_{\beta}\bm{\nabla}_{\beta} \right)^2 \bm{g} 
	&= -\bm{Q}(\bm{n} - \feq{n}) \\
	& + \delta_t \left( \bm{I}-\dfrac{\bm{Q}}{2}\right) \bm{\Gamma}
	\end{aligned}
	\label{eq:taylor_vec_2}
\end{equation}

S\'olo resta reescribir el miembro izquierdo de la \eq{eq:taylor_vec_2} para obtener una ecuaci\'on expresada completamente en el espacio de momentos. En particular, las derivadas de $\bm{g}$ pueden reescribirse como
\begin{equation}
	\begin{aligned}
		\delta_t \bm{M}\left( \dfrac{\partial}{\partial t} \bm{I} + \bm{E}_{\beta}\bm{\nabla}_{\beta} \right) \bm{g} &= \delta_t \dfrac{\partial}{\partial t}\bm{IMg} + \delta_t ( \bm{M} \bm{E}_{\beta} \bm{M}^{-1} )( \bm{M} \bm{\nabla}_{\beta} \, \bm{g}) \\
		&= \delta_t \dfrac{\partial}{\partial t}\bm{In} + \delta_t \hat{\bm{E}}_{\beta} \bm{\nabla}_{\beta} \bm{n} \\
		&= \delta_t\left( \dfrac{\partial}{\partial t}\bm{I} + \hat{\bm{E}}_{\beta} \bm{\nabla}_{\beta} \right) \bm{n}
	\end{aligned}
	\label{eq:M_propag_1}
\end{equation}
y
\begin{equation}
	\begin{aligned}
		\dfrac{\delta^2_t}{2} \bm{M}\left( \dfrac{\partial}{\partial t} \bm{I} + \bm{E}_{\beta}\bm{\nabla}_{\beta} \right)^2 \bm{g} 
		&= \dfrac{\delta^2_t}{2} \left( \dfrac{\partial}{\partial t}\bm{I} + \hat{\bm{E}}_{\beta} \bm{\nabla}_{\beta} \right) \bm{M} \left( \dfrac{\partial}{\partial t}\bm{I} + \bm{E}_{\beta} \bm{\nabla}_{\beta} \right) \bm{g} \\
		&= \dfrac{\delta^2_t}{2} \left( \dfrac{\partial}{\partial t}\bm{I} + \hat{\bm{E}}_{\beta} \bm{\nabla}_{\beta} \right) \left( \dfrac{\partial}{\partial t}\bm{I} + \hat{\bm{E}}_{\beta} \bm{\nabla}_{\beta} \right) \bm{Mg} \\
		&= \dfrac{\delta^2_t}{2} \left( \dfrac{\partial}{\partial t}\bm{I} + \hat{\bm{E}}_{\beta} \bm{\nabla}_{\beta} \right)^2 \bm{n},
	\end{aligned}
	\label{eq:M_propag_2}
\end{equation}
donde se definen las matrices $\hat{\bm{E}}_{\beta} = \bm{M} \bm{E}_{\beta} \bm{M}^{-1}$. En una grilla D2Q9, donde $\beta=x,y$, estas matrices resultan:
\begin{equation}
	\hat{\bm{E}}_{x}=
	\begin{bmatrix}
	0 & 0 & 0 & 1 & 0 & 0 & 0 & 0 & 0 \\
	0 & 0 & 0 & 1 & 1 & 0 & 0 & 0 & 0 \\
	0 & 0 & 0 & 0 & 1 & 0 & 0 & 0 & 0 \\
	2/3 & 5/3 & 0 & 0 & 0 & 0 & 0 & 1/2 & 0 \\
	0 & 1/3 & 1/3 & 0 & 0 & 0 & 0 & -1 & 0 \\
	0 & 0 & 0 & 0 & 0 & 0 & 0 & 0 & 1 \\
	0 & 0 & 0 & 0 & 0 & 0 & 0 & 0 & 1 \\
	0 & 0 & 0 & 1/3 & -1/3 & 0 & 0 & 0 & 0 \\
	0 & 0 & 0 & 0 & 0 & 2/3 & 1/3 & 0 & 0 \\
	\end{bmatrix}
\end{equation} 

\begin{equation}
	\hat{\bm{E}}_{y}=
	\begin{bmatrix}
	0 & 0 & 0 & 0 & 0 & 1 & 0 & 0 & 0 \\
	0 & 0 & 0 & 0 & 0 & 1 & 1 & 0 & 0 \\
	0 & 0 & 0 & 0 & 0 & 0 & 1 & 0 & 0 \\
	0 & 0 & 0 & 0 & 0 & 0 & 0 & 0 & 1 \\
	0 & 0 & 0 & 0 & 0 & 0 & 0 & 0 & 1 \\
	2/3 & 5/3 & 0 & 0 & 0 & 0 & 0 & 1/2 & 0 \\
	0 & 1/3 & 1/3 & 0 & 0 & 0 & 0 & 1 & 0 \\
	0 & 0 & 0 & 0 & 0 & -1/3 & 1/3 & 0 & 0 \\
	0 & 0 & 0 & 2/3 & 1/3 & 0 & 0 & 0 & 0 \\
	\end{bmatrix}
\end{equation} 

Finalmente, reemplazando las \eqs{eq:M_propag_1}{eq:M_propag_2} en la \eq{eq:taylor_vec_2}, puede obtenerse una ecuaci\'on expresada completamente en el espacio de momentos, y definida en $(\bm{x},t)$:
\begin{equation}
	\dcm \bm{n} + \dfrac{\delta_t}{2} \dcm^2 \bm{n} = -\dfrac{1}{\delta_t}\bm{Q}(\bm{n} - \feq{n}) + \delta_t \left( \bm{I}-\dfrac{\bm{Q}}{2}\right) \bm{\Gamma}.
	\label{eq:model_2d_momento}
\end{equation}

Una vez que se tiene el desarrollo en serie de Taylor de una \lbe{} en torno a $(\bm{x},t)$, es necesario aplicar el procedimiento de Chapman-Enskog, el cual consiste en la expansi\'on de las funciones de distribuci\'on y operadores diferenciales, utilizando un par\'ametro de expansi\'on $\varepsilon$:

\begin{subequations}
	\begin{equation}
		\bm{n} = \bc{n}{0} + \varepsilon \bc{n}{1} + \varepsilon^2 \bc{n}{2} + \cdots = \sum_{k=0}^{\infty} \varepsilon^k \bc{n}{k}
		\label{eq:n_chapman}
	\end{equation}
	\begin{equation}
		\dfrac{\partial}{\partial t} = \varepsilon \dfrac{\partial}{\partial t_1} + 	\varepsilon^2 \dfrac{\partial}{\partial t_2}
	\end{equation}
	\begin{equation}
		\dfrac{\partial}{\partial x_{\beta}} = \varepsilon \dfrac{\partial}{\partial x_{\beta_1}}
	\end{equation}
	\begin{equation}
		\bm{\Gamma} = \varepsilon \bc{\Gamma}{1}
		\label{eq:gamma_chapman}
	\end{equation}
	\label{eq:escalas_che}
\end{subequations}

Las expansiones determinadas por las \eqto{eq:n_chapman}{eq:gamma_chapman} pueden incorporarse a la \eq{eq:model_2d_momento}, y si se desprecian los t\'erminos de orden superior a $\varepsilon^2$ se obtiene:
\begin{equation}
	\begin{aligned}
		\varepsilon \dcmuno \bc{n}{0} + \varepsilon^2 \dcmuno \bc{n}{1} + \varepsilon^2 \dfrac{\partial}{\partial t_2} \bm{I} \bc{n}{0} \\
		+ \varepsilon^2 \dfrac{\delta_t}{2} \dcmuno^2 \bc{n}{0} = -\dfrac{1}{\delta_t} \bm{Q} \left( \bc{n}{0} + \varepsilon \bc{n}{1} + \varepsilon^2 \bc{n}{2} - \feq{n} +  \right) \\
		+ \varepsilon \left( \bm{I}-\dfrac{\bm{Q}}{2}\right) \bc{\Gamma}{1}
	\end{aligned}
	\label{eq:modelo_2d_escalas}
\end{equation}

De esta manera, con las variables y operadores diferenciales reemplazados por las expansiones correspondientes, puede generarse un conjunto de ecuaciones (una para cada potencia de $\varepsilon$) a partir de la separaci\'on de la \eq{eq:modelo_2d_escalas}:

\begin{subequations}
	\begin{align}
		\varepsilon^0: && \bc{n}{0} = \feq{n}\\
		\varepsilon^1: && \dcmuno \bc{n}{0} = -\dfrac{1}{\delta_t} \bm{Q} \bc{n}{1} + \left( \bm{I}-\dfrac{\bm{Q}}{2}\right) \bc{\Gamma}{1} \label{eq:eps_1_0}\\	
		\varepsilon^2: && {\scriptstyle \dcmuno \bc{n}{1} + \dfrac{\partial \bc{n}{0}}{\partial t_2}  + \dfrac{\delta_t}{2} \dcmuno^2 \bc{n}{0}  = -\dfrac{1}{\delta_t} \bm{Q} \bc{n}{2}} \label{eq:eps_2_0}
	\end{align}
\end{subequations}


Usando la \eq{eq:eps_1_0} para reescribir el t\'ermino cuadr\'atico de la \eq{eq:eps_2_0}, y reagrupando t\'erminos, puede escribirse el conjunto de ecuaciones correspondientes a cada escala de expansi\'on:
\begin{subequations}
	\begin{align}
		\varepsilon^0: && \bc{n}{0} = \feq{n} \label{eq:eps_0}\\
		\varepsilon^1: && \dcmuno \bc{n}{0} - \bc{\Gamma}{1} = -\dfrac{1}{\delta_t} \bm{Q} \left( \bc{n}{1} + \dfrac{\delta_t}{2} \bc{\Gamma}{1} \right)  \label{eq:eps_1}\\
		\varepsilon^2: && \dcmuno \left( \bm{I}-\dfrac{\bm{Q}}{2}\right) \left( \bc{n}{1} + \dfrac{\delta_t}{2} \bc{\Gamma}{1} \right) + \dfrac{\partial \bc{n}{0}}{\partial t_2}  =  -\dfrac{1}{\delta_t} \bm{Q} \bc{n}{2} \label{eq:eps_2}
	\end{align}
\end{subequations}

El reordenamiento de t\'erminos utilizado en la construcci\'on de las \eqto{eq:eps_0}{eq:eps_2} permite visualizar expl\'icitamente el v\'inculo resultante entre cada escala: el t\'ermino de fuente de cada escala (en este caso, el miembro derecho), corresponde a la variable transportada en la escala de orden inmediatamente superior.

Hasta este punto, el an\'alisis sirve para hacer expl\'icito el proceso de separaci\'on de escalas determinado por la expansi\'on de Chapman-Enskog, desarrollado con el objetivo de encontrar la ecuaci\'on recuperada a partir de la \eq{eq:g_model_2d}. Sin embargo, para finalizar el proceso es necesario definir a la distribuci\'on de equilibrio y la matriz de coeficientes de relajaci\'on, dado que todav\'ia no se hizo uso de una forma expl\'icita de las mismas. Siguiendo con las ideas expuestas en los trabajos de Huang y sus colaboradores, la recuperaci\'on de una ecuaci\'on macrosc\'opica escalar reduce las restricciones de isotrop\'ia sobre el conjunto de velocidades. Adem\'as, como no es necesario recuperar t\'erminos del tipo $\nabla \cdot(\rho \bm{uu})$, entonces es factible construir una distribuci\'on de equilibrio que s\'olo contenga t\'erminos hasta primer orden en la velocidad del fluido. Como esta distribuci\'on es utilizada dentro de un esquema de colisi\'on MRT, entonces resulta m\'as sencillo si se construye directamente en el espacio de momentos:
\begin{equation}
	\feq{n} = T(1, \alpha_1, \alpha_2, \alpha_3 u_x, \alpha_4 u_x, \alpha_5 u_y, \alpha_6 u_y, 0, 0 )^T.
\end{equation}
En este caso, las constantes $\{ \alpha_i \}$ constituyen un conjunto de par\'ametros libres que ser\'an empleados para determinar propiedades de la ecuaci\'on recuperada. Esta nueva distribuci\'on de equilibrio preserva el orden de los momentos de grilla extablecidos por $\bm{M}$, pero a diferencia de la \eq{eq:feq_li}, no se consideran los t\'erminos cuadr\'aticos de la velocidad del fluido.

Por otro lado, la \'unica restricci\'on impuesta hasta el momento sobre los coeficientes de la matriz $\bm{Q}$ es que sean constantes. En los operadores MRT tradicionales esta matriz es diagonal, y cada coeficiente se asocia a la relajaci\'on individual del momento correspondiente. Sin embargo, en las \eqs{eq:eps_1}{eq:eps_2} es posible ver que si $\bm{Q}$ tambi\'en contiene elementos no nulos fuera de la diagonal principal, se logra que las ecuaciones para cada componente incorporen tambi\'en otros momentos dentro de la misma escala de $\varepsilon$. De esta forma, con una elecci\'on de coeficientes adecuada se podr\'ian cancelar t\'erminos macrosc\'opicos no deseados. Para este caso, se propone que $\bm{Q}$ sea de la forma:

\begin{equation}
	\bm{Q}=
	\begin{bmatrix}
	q_0 & 0 & 0 & 0 & 0 & 0 & 0 & 0 & 0 \\
	0 & q_1 & 0 & 0 & 0 & 0 & 0 & 0 & 0 \\
	0 & 0 & q_2 & 0 & 0 & 0 & 0 & 0 & 0 \\
	0 & 0 & 0 & q_3 & q_{34} & 0 & 0 & 0 & 0 \\
	0 & 0 & 0 & 0 & q_4 & 0 & 0 & 0 & 0 \\
	0 & 0 & 0 & 0 & 0 & q_5 & q_{56} & 0 & 0 \\
	0 & 0 & 0 & 0 & 0 & 0 & q_6 & 0 & 0 \\
	0 & 0 & 0 & 0 & 0 & 0 & 0 & q_7 & 0 \\
	0 & 0 & 0 & 0 & 0 & 0 & 0 & 0 & q_8 \\
	\end{bmatrix},
\end{equation} 
donde los coeficientes $q_{34}$ y $q_{56}$ ser\'an definidos convenientemente.  

Con la definici\'on expl\'icita de $\feq{n}$ y $\bm{Q}$ es posible completar el an\'alisis de expansi\'on multiescala. En particular es necesario evaluar, dentro de cada escala, aquellos componentes correspondientes a las cantidades conservadas. Como en este caso s\'olo se tiene $T$, se necesitan \'unicamente los primeros componentes (\'indice 0). Para la escala $\varepsilon$, de la \eq{eq:eps_1} se tiene:
\begin{equation}
	\dfrac{ \partial \nce{0}{0}}{\partial t_1}  +  \dfrac{ \partial \nce{3}{0}}{\partial x_1} + \dfrac{ \partial \nce{5}{0}}{\partial y_1} - \Gamma_0^{(1)}= -\dfrac{q_0}{\delta_t} \left( \nce{0}{1} + \dfrac{\delta_t}{2} \Gamma_0^{(1)} \right),
\end{equation}

\begin{equation}
	\dfrac{ \partial T}{\partial t_1}  +  \dfrac{ \partial (\alpha_3 T u_x)}{\partial x_1} + \dfrac{ \partial (\alpha_5 T u_y)}{\partial y_1} = -\dfrac{q_0}{\delta_t} \left( \nce{0}{1} + \dfrac{\delta_t}{2} \Gamma_0^{(1)} \right) + \Gamma_0^{(1)}.
	\label{eq:T_eps_1}
\end{equation}

Como
\begin{equation}
	T = \sum_{\alpha} g_{\alpha} + \dfrac{\delta_t}{2} \Gamma_0 = n_0 + \dfrac{\delta_t}{2} \Gamma_0
\end{equation}
y por las \eqs{eq:n_chapman}{eq:eps_0} se cumple:
\begin{equation}
	n_0 = \nce{0}{0} + \varepsilon \nce{0}{1} + \varepsilon^2 \nce{0}{2} + \cdots = n_0^{eq} + \varepsilon \nce{0}{1} + \varepsilon^2 \nce{0}{2} + \cdots,
\end{equation}
entonces dentro de cada escala se satisface:
\begin{equation}
	\begin{aligned}
		\varepsilon^0: && \nce{0}{0} = n_0^{eq} \\
		\varepsilon^1: && \nce{0}{1} + \dfrac{\delta_t}{2} \Gamma_0^{(1)} = 0\\
		\varepsilon^2: && \nce{0}{2} = 0 \\
		&\vdots& \\
		\varepsilon^k: && \nce{0}{k} = 0 \quad \forall \, k > 2.
	\end{aligned}
	\label{eq:ncero_che}
\end{equation}

Por lo tanto, si se usa $\alpha_3 = \alpha_5 = 1$, la \eq{eq:T_eps_1} finalmente se reduce a:
\begin{equation}
	\dfrac{\partial T}{\partial t_1} + \nabla_1 \cdot (T\bm{u}) = \Gamma_0^{(1)}.
	\label{eq:T_eps_1_macro}
\end{equation}

Por otro lado, para la escala $\varepsilon^2$ se necesita la primera componente de la \eq{eq:eps_2}:
\begin{equation}
	\begin{split}
	\dfrac{\partial \nce{0}{0}}{\partial t_2} + \dfrac{\partial}{\partial t_1}\left( 1-\dfrac{q_0}{2} \right) \left( \nce{0}{1} +\dfrac{\delta_t}{2} \Gamma_0^{(1)}\right) \\
	+ \dfrac{\partial}{\partial x_1}\left[ \left( 1-\dfrac{q_3}{2} \right) \left( \nce{3}{1} +\dfrac{\delta_t}{2} \Gamma_3^{(1)}\right) - \dfrac{q_{34}}{2} \left( \nce{4}{1} +\dfrac{\delta_t}{2} \Gamma_4^{(1)}\right) \right] \\
	+ \dfrac{\partial}{\partial y_1}\left[ \left( 1-\dfrac{q_5}{2} \right) \left( \nce{5}{1} +\dfrac{\delta_t}{2} \Gamma_5^{(1)}\right) - \dfrac{q_{56}}{2} \left( \nce{6}{1} +\dfrac{\delta_t}{2} \Gamma_6^{(1)}\right) \right] = -\dfrac{1}{\delta_t} q_0 \nce{0}{2}
	\end{split}
	\label{eq:eps_2_comp_0}
\end{equation}

En la \eq{eq:eps_2_comp_0} aparecen diferentes componentes del t\'ermino de fuente $\bm{\Gamma}$. En esta etapa, se propone el uso de una fuente sencilla con una \'unica componente no nula, es decir:
\begin{equation}
	\bm{\Gamma} = (s,0,0,0,0,0,0,0,0)^T.	
\end{equation}

De esta manera, como $\Gamma = \varepsilon \Gamma^{(1)}$, entonces la \'unica componente no nula es $\Gamma_0^{(1)}$. Usando adem\'as a la \eq{eq:ncero_che}, puede reescribirse a la \eq{eq:eps_2_comp_0} como:
\begin{equation}
	\dfrac{\partial T}{\partial t_2} + 
	\underbrace{\dfrac{\partial}{\partial x_1}\left[ \left( 1-\dfrac{q_3}{2} \right) \nce{3}{1} - \dfrac{q_{34}}{2} \nce{4}{1} \right]}_{I} 	+  \underbrace{ \dfrac{\partial}{\partial y_1}\left[ \left( 1-\dfrac{q_5}{2} \right) \nce{5}{1} - \dfrac{q_{56}}{2}  \nce{6}{1} \right]}_{II} = 0	
	\label{eq:eps_2_reduced}
\end{equation}

Como se observa en la \eq{eq:eps_2_reduced}, son necesarias ecuaciones para $\nce{3}{1}$, $\nce{4}{1}$, $\nce{5}{1}$ y $\nce{6}{1}$, y que pueden obtenerse de la \eq{eq:eps_2}. Por ejemplo, para $\nce{3}{1}$ se tiene:
\begin{equation}
	\dfrac{\partial \nce{3}{0}}{\partial t_1} + \dfrac{2}{3}\dfrac{\partial \nce{0}{0}}{\partial x_1} + \dfrac{1}{6}\dfrac{\partial \nce{1}{0}}{\partial x_1} - \Gamma_3^{(1)} = -\dfrac{q_3}{\delta_t} \left( \nce{3}{1} +\dfrac{\delta_t}{2} \Gamma_3^{(1)}\right) - \dfrac{q_{34}}{\delta_t} \left( \nce{4}{1} +\dfrac{\delta_t}{2} \Gamma_4^{(1)}\right)
\end{equation}
y reemplazando componentes:
\begin{equation}
	\dfrac{\partial (Tu_x)}{\partial t_1} + \dfrac{\partial}{\partial x_1} \left(\dfrac{4+\alpha_1}{6} T \right) = -\dfrac{q_3}{\delta_t} \nce{3}{1} - \dfrac{q_{34}}{\delta_t} \nce{4}{1}.
	\label{eq:n3_2d}
\end{equation}

Por otro lado, para $\nce{4}{1}$ se obtiene \red{revisar}:
\begin{equation}
	\dfrac{\partial \nce{4}{0}}{\partial t_1} + \dfrac{1}{3}\dfrac{\partial \nce{1}{0}}{\partial x_1} + \dfrac{1}{1}\dfrac{\partial \nce{2}{0}}{\partial x_1}  + \dfrac{\partial \nce{8}{0}}{\partial y_1} - \dfrac{\partial \nce{7}{0}}{\partial x_1}	
	= -\dfrac{q_4}{\delta_t} \left( \nce{4}{1} +\dfrac{\delta_t}{2} \Gamma_4^{(1)}\right)
\end{equation}
\begin{equation}
	\dfrac{\partial (\alpha_4 Tu_x)}{\partial t_1} + \dfrac{\partial}{\partial x_1} \left(\dfrac{\alpha_1 + \alpha_2}{3} T \right) = -\dfrac{q_4}{\delta_t} \nce{4}{1} 
	\label{eq:n4_2d}
\end{equation}

Las ecuaciones \eqs{eq:n3_2d}{eq:n4_2d} pueden usarse para despejar $\nce{3}{1}$ y $\nce{4}{1}$, y poder as\'i reescribir el t\'ermino $I$ de la \eq{eq:eps_2_reduced}. Si se reemplazan dichas componentes, considerando $\alpha_4 = -1$, $I$ puede escribirse como:
\begin{equation}
\begin{aligned}
	I =& -\delta_t \left( \dfrac{1}{q_3} - \dfrac{1}{2}  + \dfrac{q_{34}}{q_3 q_4}  \right) \dfrac{\partial (Tu_x)}{\partial t_1} - \delta_t \left( \dfrac{1}{q_3} - \dfrac{1}{2} \right) \dfrac{\partial}{\partial x_1} \left(\dfrac{4+\alpha_1}{6} T \right) \\
	&+ \dfrac{q_{34}\delta_t}{q_3q_4}\dfrac{\partial}{\partial x_1} \left(\dfrac{\alpha_1+\alpha_2}{3} T \right)
	\label{eq:I_0}
\end{aligned}
\end{equation}


En la \eq{eq:I_0} puede apreciarse la utilidad de considerar una matriz $\bm{Q}$ con elementos no nulos fuera de la diagonal, ya que estos elementos pueden elegirse convenientemente para eliminar t\'erminos no deseados en las ecuaciones recuperadas. En particular, el t\'ermino con $\partial_{t_1}(Tu_x)$ puede cancelarse si se anula la combinacion de factores de relajaci\'on que lo multiplican, es decir:
\begin{equation}
	q_{34} = q_4 \left( \dfrac{q_3}{2} - 1 \right).
\end{equation}

Con esta elecci\'on, la \eq{eq:I_0} finalmente puede reescribirse como:
\begin{equation}
	I = -\delta_t \left( \dfrac{1}{q_3} - \dfrac{1}{2} \right) \dfrac{\partial}{\partial x_1} \left(\dfrac{4+3\alpha_1 + 2\alpha_2}{6} T \right)
	\label{eq:I_1}
\end{equation}

Este resultado es fundamental para el desarrollo de modelos que busquen recuperar ecuaciones de advecci\'on-difusi\'on para variables escalares, ya que puede demostrarse formalmente que una elecci\'on adecuada de los coeficientes extra-diagonales de la matriz de relajaci\'on, lleva a la cancelaci\'on de aquellos t\'erminos no deseados en la ecuaci\'on macrosc\'opica. En este caso, con la definici\'on correcta de $q_{34}$ es posible eliminar $\partial_{t_1}(Tu_x)$ en las escalas de expansi\'on analizadas.

El mismo procedimiento puede realizarse para reescribir el tçérmino $II$ de la \eq{eq:eps_2_reduced}: despejando $\nce{5}{1}$ y $\nce{6}{1}$ de las componentes 5 y 6 de la \eq{eq:eps_2}, y usando $\alpha_6=-1$, $II$ puede reescribirse como:
\begin{equation}
\begin{aligned}
	II =& -\delta_t \left( \dfrac{1}{q_5} - \dfrac{1}{2}  + \dfrac{q_{56}}{q_5 q_6}  \right) \dfrac{\partial (Tu_y)}{\partial t_1} - \delta_t \left( \dfrac{1}{q_5} - \dfrac{1}{2} \right) \dfrac{\partial}{\partial y_1} \left(\dfrac{4+\alpha_1}{6} T \right) \\
	&+ \dfrac{q_{56}\delta_t}{q_5q_6}\dfrac{\partial}{\partial y_1} \left(\dfrac{\alpha_1+\alpha_2}{3} T \right)
	\label{eq:II_0}
\end{aligned}
\end{equation}


Por lo tanto, si 
\begin{equation}
	q_{56} = q_6 \left( \dfrac{q_5}{2} - 1 \right),
\end{equation}
entonces la \eq{eq:II_0} se reduce a:
\begin{equation}
	II = -\delta_t \left( \dfrac{1}{q_5} - \dfrac{1}{2} \right) \dfrac{\partial}{\partial y_1} \left(\dfrac{4+3\alpha_1 + 2\alpha_2}{6} T \right).
	\label{eq:II_1}
\end{equation}

Las \eqs{eq:I_1}{eq:II_1} pueden finalmente utilizarse para completar el an\'alisis en la escala $\varepsilon^2$:
\begin{equation}
	\begin{split}
	\dfrac{\partial T}{\partial t_2} -
	\dfrac{\partial}{\partial x_1}\left[ \delta_t \left( \dfrac{1}{q_3} - \dfrac{1}{2} \right) \dfrac{\partial}{\partial x_1} \left(\dfrac{4+3\alpha_1 + 2\alpha_2}{6} T \right) \right]	\\
	-  \dfrac{\partial}{\partial y_1}\left[ \delta_t \left( \dfrac{1}{q_5} - \dfrac{1}{2} \right) \dfrac{\partial}{\partial y_1} \left(\dfrac{4+3\alpha_1 + 2\alpha_2}{6} T \right) \right] = 0.
	\end{split}
	\label{eq:eps_2_clean}
\end{equation}

En particular, si adem\'as se impone que $q_3=q_5=q_{\chi}$, entonces la \eq{eq:eps_2_clean} puede reescribirse como:

\begin{equation}
	\dfrac{\partial T}{\partial t_2} - \nabla_1 \cdot \left[ \delta_t \left( \dfrac{1}{q_{\chi}} - \dfrac{1}{2} \right) \, \nabla\left( \dfrac{4+3\alpha_1 + 2\alpha_2}{6} T \right) \right]  = 0.
	\label{eq:T_eps_2}
\end{equation}

Esta parte del an\'alisis de Chapman-Enskog implic\'o encontrar la descripci\'on del comportamiento de las variables macrosc\'opicas en las escalas de $\varepsilon$ propuestas, de modo que s\'olo resta volver a las escalas originales. Por lo tanto, multiplicando por $\varepsilon$ a la \eq{eq:T_eps_1_macro}, por $\varepsilon^2$ a la \eq{eq:T_eps_2} y sum\'andolas, se tiene:
\begin{equation}
	\varepsilon \dfrac{\partial T}{\partial t_1} + \varepsilon^2 \dfrac{\partial T}{\partial t_2} + \varepsilon \nabla_1 \cdot (T\bm{u}) - \varepsilon^2 \nabla_1 \cdot (\chi \nabla T) - \varepsilon \Gamma_0^{(1)} = 0,
\end{equation}
donde se define a la difusividad t\'ermica $\chi$ como:
\begin{equation}
	\chi = \delta_t \left( \dfrac{1}{q_{\chi}} - \dfrac{1}{2} \right) \left( \dfrac{4+3\alpha_1 + 2\alpha_2}{6} \right).
	\label{eq:modelo_2d_chi}
\end{equation}

Usando la expansi\'on de escalas dada por la \eq{eq:escalas_che}, la ecuaci\'on macroc\'opica recuperada finalmente resulta:
\begin{equation}
	\dfrac{\partial T}{\partial t} + \nabla \cdot (T\bm{u}) = \nabla \cdot (\chi \nabla T) + s.
	\label{eq:T_2d}
\end{equation}

El procedimiento dessarrollado para mostrar que la \eq{eq:g_model_2d} produce una temperatura macrosc\'opica que satisface la \eq{eq:T_2d}, resalta propiedades fundamentales de la propuesta. Por un lado, la construcci\'on \emph{ad-hoc} de una distribuci\'on de equilibrio directamente en el espacio de momentos, de una matriz de relajaci\'on con coeficientes no nulos fuera de la diagonal, y de un t\'ermino de fuente definido directamente en el espacio de momentos, permite recuperar una ecuaci\'on de advecci\'on-difusi\'on para $T$ sin los t\'erminos no deseados caracter\'isticos de esquemas tradicionales. Por otro lado, la nueva distribuci\'on de equilibrio $\feq{n}$ contiene dos par\'ametros libres, $\alpha_1$ y $\alpha_2$, que permiten ajustar la difusividad t\'ermica recuperada sin necesidad de modificar al coeficiente de relajaci\'on $q_{\chi}$. Esta flexibilidad permite, en principio, mejorar los l\'imites de estabilidad del modelo.

La \eq{eq:T_2d} es similar a la ecuaci\'on objetivo de M\'arkus y H\'azi (\eq{eq:markus}), aunque el t\'ermino de fuente $s$ debe ser definido correctamente para incorporar expl\'icitamente los t\'erminos faltantes. En definitiva, reuniendo las propuestas y suposiciones realizadas, es posible expresar lo siguiente a modo de \textbf{s\'intesis} del modelo: la ecuaci\'on diferencial para $T$
\begin{equation}
	\dfrac{\partial T}{\partial t} + \nabla \cdot (\bm{u} T) = \chi \nabla^2 T  + \dfrac{\chi}{\rho} \nabla T \cdot \nabla \rho + T \left[ 1 - \dfrac{1}{\rho c_v} \left( \dfrac{\partial p_{EOS}}{\partial T} \right)_{\rho} \right] \nabla \cdot \bm{u}
\end{equation}
puede recuperarse usando una funci\'on de distribuci\'on $\bm{g}$ que satisfaga la siguiente \lbe{} en el espacio de momentos:
\begin{subequations}
	\begin{equation}
		\bm{g}(\bm{x} + \bm{e} \delta_t, t+\delta_t) = \bm{M}^{-1} \left[ \bm{n} - \bm{Q}	(\bm{n} - \feq{n}) + \delta_t \left( \bm{I}-\dfrac{\bm{Q}}{2}\right) \bm{\Gamma}\right],		
	\end{equation}
	\begin{equation}
		\bm{n} = \bm{Mg}
	\end{equation}
	\begin{equation}
		T = \sum_{\alpha} g_{\alpha} + \dfrac{\delta_t}{2} \Gamma_0,		
	\end{equation}
	\begin{equation}
		\feq{n} = T(1, \alpha_1, \alpha_2, u_x, -u_x, u_y, -u_y, 0, 0 )^T,
	\end{equation}
	\begin{equation}
		\bm{\Gamma} = (s,0,0,0,0,0,0,0,0)^T,		
	\end{equation}
	\begin{equation}
		s = \dfrac{\chi}{\rho} \nabla T \cdot \nabla \rho + T \left[ 1 - \dfrac{1}{\rho c_v} \left( \dfrac{\partial p_{EOS}}{\partial T} \right)_{\rho} \right] \nabla \cdot \bm{u}
		\label{eq:model2D_hs}
	\end{equation}
	\begin{equation}
	 	\bm{Q}=
		\begin{bmatrix}
		q_0 & 0 & 0 & 0 & 0 & 0 & 0 & 0 & 0 \\
		0 & q_1 & 0 & 0 & 0 & 0 & 0 & 0 & 0 \\
		0 & 0 & q_2 & 0 & 0 & 0 & 0 & 0 & 0 \\
		0 & 0 & 0 & q_{\chi} & q_{34} & 0 & 0 & 0 & 0 \\
		0 & 0 & 0 & 0 & q_4 & 0 & 0 & 0 & 0 \\
		0 & 0 & 0 & 0 & 0 & q_{\chi} & q_{56} & 0 & 0 \\
		0 & 0 & 0 & 0 & 0 & 0 & q_6 & 0 & 0 \\
		0 & 0 & 0 & 0 & 0 & 0 & 0 & q_7 & 0 \\
		0 & 0 & 0 & 0 & 0 & 0 & 0 & 0 & q_8 \\
		\end{bmatrix},	
	\end{equation}
	\begin{equation}
		q_{34} = q_4 \left( \dfrac{q_{\chi}}{2} - 1 \right),
	\end{equation}
	\begin{equation}
		q_{56} = q_6 \left( \dfrac{q_{\chi}}{2} - 1 \right).
	\end{equation}
	\label{eq:modelo_2d_full}
\end{subequations}


\red{Poner que se resuelve todo de manera acoplada con la ecuaci\'on de Li.}

\subsubsection{C\'alculo del t\'ermino de fuente}

Con la definici\'on del t\'ermino de fuente dado por la \eq{eq:model2D_hs}, es necesario calcular expl\'icitamente las derivadas espaciales de las principales variables macrosc\'opicas. Esto puede realizarse, por ejemplo, usando esquemas de dferencias finitas centradas de segundo orden. Alternativamente es posible aprovechar las propiedades de la grilla mediante el uso de esquemas isotr\'opicos de segundo orden \cite{lee_stable_2005}:
\begin{equation}
	\dfrac{\partial \phi}{\partial \beta} \approx \dfrac{1}{c_s^2 \delta_t} \sum_{\alpha} \omega_{\alpha} \phi(\bm{x} + \bm{e}_{\alpha}\delta_t)e_{\alpha\beta}
\end{equation}

Esta aproximaci\'on, si bien es sencilla y natural para la formulaci\'on de lattice Boltzmann, muchas veces es indeseada ya que rompe (como los esquemas de diferencias est\'andar) la localidad del proceso de colisi\'on. Esta caracter\'istica toma relevancia cuando los modelos son implementados con t\'ecnicas de procesamiento distribuido. Una soluci\'on intermedia a este problema radica en aprovechar los valores de no equuilibrio de las funciones de distribuci\'on para calcular el gradiente de las cantidades conservadas. En particular, para encontrar una expresi\'on adecuada de $\nabla T$ podemos sumar las \eqs{eq:n3_2d}{eq:n4_2d} para obtener:
\begin{equation}
\begin{aligned}
	\dfrac{\partial}{\partial x_1} \left( \dfrac{4+3\alpha_1+2\alpha_2}{6} T \right) &= -\dfrac{q_3}{\delta_t} \nce{3}{1} - \dfrac{q_{34}}{\delta_t} \nce{4}{1} - \dfrac{q_4}{\delta_t} \nce{4}{1} \\
	&= -\dfrac{q_3}{\delta_t} \nce{3}{1} - \dfrac{q_3 q_4}{a \delta_t} \nce{4}{1}
\end{aligned}
	\label{eq:partial_1_x}
\end{equation}

De la misma manera, combinando las ecuaciones relevantes para $\nce{5}{1}$ y $\nce{6}{1}$ resulta:
\begin{equation}
\begin{aligned}
	\dfrac{\partial}{\partial y_1} \left( \dfrac{4+3\alpha_1+2\alpha_2}{6} T \right) &= -\dfrac{q_5}{\delta_t} \nce{5}{1} - \dfrac{q_{56}}{\delta_t} \nce{6}{1} - \dfrac{q_6}{\delta_t} \nce{6}{1} \\
	&= -\dfrac{q_5}{\delta_t} \nce{5}{1} - \dfrac{q_5 q_6}{a \delta_t} \nce{6}{1}
\end{aligned}
	\label{eq:partial_1_y}
\end{equation}

Por lo tanto, combinando las \eqs{eq:partial_1_x}{eq:partial_1_y} se tiene:
\begin{equation}
	\left( \dfrac{4+3\alpha_1+2\alpha_2}{6} \right) \nabla_1 T= -\dfrac{1}{\delta_t}
	\left[ 
 	\begin{array}{c} 
 		q_3 \nce{3}{1}  +  \dfrac{q_3q_4}{2}\nce{4}{1} \\[2mm]
 		q_5 \nce{5}{1}  +  \dfrac{q_5q_6}{2}\nce{6}{1}
 	\end{array} 
	\right].
	\label{eq:grad1_T}
\end{equation}

Si se considera una expansi\'on de $\bm{n}$ hasta $\varepsilon$
\begin{equation}
	\varepsilon n_i^{(1)} = n_i - n_i^{(0)}
\end{equation}
entonces, multiplicando a la \eq{eq:grad1_T} por $\varepsilon$ y reagrupando t\'erminos se obtiene una expresi\'on para el gradiente de la temperatura:
\begin{equation}
	\nabla T= -\dfrac{1}{\delta_t}\left( \dfrac{6}{4+3\alpha_1+2\alpha_2} \right)
	\left[ 
 	\begin{array}{c} 
 		q_3 (n_3 - n_3^{eq})  +  \dfrac{q_3q_4}{2} (n_4 - n_4^{eq}) \\[2mm]
 		q_5 (n_5 - n_5^{eq})  +  \dfrac{q_5q_6}{2} (n_6 - n_6^{eq})
 	\end{array} 
	\right],
	\label{eq:gradT_2d}
\end{equation}
donde se us\'o $n_i^{(0)} = n_i^{eq}$. De esta manera, la \eq{eq:gradT_2d} provee una aproximaci\'on para $\nabla T$ en cada nodo que no requiere conocer los valores de $T$ en los vecinos, sino que hace uso de las desviaciones de $\bm{n}$ respecto a la condici\'on de equilibrio.

Los resultados del proceso de validaci\'on presentados en las secciones siguientes fueron realizados con el objetivo de cuantificar la precisi\'on del modelo propuesto. Por lo tanto, salvo que se indique lo contrario, las simulaciones fueron efectuadas utilizando un esquema de diferencias centradas de segundo orden para $\nabla T$, con el fin de evitar posibles efectos adicionales originados por la \eq{eq:gradT_2d}. Los efectos de esta aproximaci\'on son analizados expl\'icitamente en la \red{Secci\'on}.



\input{Transferencia_de_calor_2d_validacion.tex}