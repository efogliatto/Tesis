\chapter{Transferencia de calor en flujo multif\'asico bidimensional}

En general, los modelos existentes de lattice Boltzmann para simular transferencia de calor en flujos multif\'asicos pueden clasificarse en tres categor\'ias principales: modelos de velocidad m\'ultiple (\emph{multi-speed}), modelos h\'ibridos (\emph{hybrid}) y modelos con dos funciones de distribuci\'on (\emph{double-distribution}). En el primer caso, las propiedades macrosc\'opicas del fluido como densidad, velocidad y temperatura, se obtienen a partir de los momentos cero, primero y segundo de una \'unica funci\'on de distribuci\'on \cite{alexander_lattice_1993,succi_lattice_2018}. A pesar que estos modelos cuentan con un s\'olido fundamento te\'orico, su uso difusi\'on es limitada, ya que t\'ipicamente emplean un set de velocidades mayor que el correspondiente modelo isot\'ermico. suelen incorporar t\'erminos de mayor orden en la funci\'on de distribuci\'on de equilibrio, y est\'an restringidos a la simulaci\'on de un fluido con n\'umero de Prandtl fijo.

Estas limitaciones intr\'insecas de los modelos de velocidad m\'ultiple motivaron el desarrollo de otro tipo de modelos t\'ermicos, donde se introduce un acoplamiento expl\'icito entre un esquema isot\'ermico para las ecuaciones hidrodin\'amicas, y una t\'ecnica alternativa para el transporte de energ\'ia.