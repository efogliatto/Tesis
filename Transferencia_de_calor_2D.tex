\chapter{Transferencia de calor en flujo multif\'asico bidimensional}

En general, los modelos existentes de lattice Boltzmann para simular transferencia de calor en flujos multif\'asicos pueden clasificarse en tres categor\'ias principales: modelos de velocidad m\'ultiple (\emph{multi-speed}), modelos h\'ibridos (\emph{hybrid}) y modelos con dos funciones de distribuci\'on (\emph{double-distribution}). En el primer caso, las propiedades macrosc\'opicas del fluido como densidad, velocidad y temperatura, se obtienen a partir de los momentos cero, primero y segundo de una \'unica funci\'on de distribuci\'on \cite{alexander_lattice_1993,succi_lattice_2018}. A pesar que estos modelos cuentan con un s\'olido fundamento te\'orico, su uso difusi\'on es limitada, ya que t\'ipicamente emplean un set de velocidades mayor que el correspondiente modelo isot\'ermico. suelen incorporar t\'erminos de mayor orden en la funci\'on de distribuci\'on de equilibrio, y est\'an restringidos a la simulaci\'on de un fluido con n\'umero de Prandtl fijo.

Estas limitaciones intr\'insecas de los modelos de velocidad m\'ultiple motivaron el desarrollo de otro tipo de modelos t\'ermicos, donde se introduce un acoplamiento expl\'icito entre un esquema isot\'ermico para las ecuaciones hidrodin\'amicas, y una t\'ecnica alternativa para el transporte de energ\'ia. Los modelos h\'ibridos \cite{dong_numerical_2012,li_lattice_2015}, por un lado, aprovechan la discretizaci\'on del dominio asociada a la ecuaci\'on hidrodin\'amica, y hacen uso de un esquema c\'asico de diferencias finitas para encontrar la soluci\'on de la ecuaci\'on de energ\'ia correspondiente. Por otro lado, los modelos con dos funciones de distribuci\'on introducen una segunda \lbe{} para recuperar la ecuaci\'on macrosc\'opica deseada.

Los esquemas de doble funci\'on de distribuci\'on heredan las principales ventajas de lattice Boltzmann, como simplicidad del algoritmo y elevada eficiencia computacional. Sin embargo, tambi\'en presentan las limitaciones comunes derivadas de la representaci\'on de ecuaciones escalares de advecci\'on-difusi\'on \cite{markus_simulation_2011, huang_modified_2014, li_improved_2017, huang_numerical_2011, li_effect_2014, huang_multiphase_2015}. En este tipo de ecuaciones hay una \'unica cantidad conservada (por ejemplo temperatura), de modo que se relajan las condiciones de isotrop\'ia y es posible, en teor\'ia, emplear distribuciones de equilibrio lineales (en la velocidad del fluido) o conjuntos de velocidades reducidos. Sin embargo, la expansi\'on de Chapman-Enskog de estos esquemas muestra que existen t\'erminos no deseados que se recuperan inevitablemente en las ecuaciones macr\'oscipicas \cite{huang_modified_2014, huang_numerical_2011, kruger_lattice_2017}, y en el caso de los modelos para flujo multif\'asico con transferencia de calor, estos t\'erminos producen fuentes de calor adicionales que dependen del potencial de interacci\'on o de derivadas temporales de la velocidad del fluido.

Un ejemplo t\'ipico de estas restricciones puede analizarse con el uso de una \lbe{} con operador de colisi\'on SRT destinada a recuperar una ecuaci\'on de advecci\'on difusi\'on de un escalar pasivo $\phi$: \red{D2Q9?}
\begin{equation}
	\dfrac{\partial \phi}{\partial t} + \nabla \cdot (\phi \bm{u}) = \nabla \cdot (k \nabla \phi),
	\label{eq:adv_dif_phi}
\end{equation}
donde $k$ es una constante de difusi\'on. En este caso, si se utiliza un esquema cl\'asico de lattice Boltzmann, es decir:
\begin{equation}
	\begin{gathered}
		g_{\alpha}(\bm{x}+\bm{e}_{\alpha}\delta_t, t+\delta_t) - g_{\alpha}(\bm{x},t) = -\dfrac{1}{\tau} \left[ g_{\alpha}(\bm{x},t) - g^{eq}_{\alpha}(\bm{x},t) \right], \\[2mm]
		g_{\alpha}^{eq} = w_i \phi \left[ 1 + \dfrac{e_{i\alpha}u_{\alpha}}{c_s^2} + \dfrac{u_{\alpha}u_{\beta}(e_{i\alpha}e_{i\beta}-c_s^2\delta_{\alpha\beta})}{2c_s^4} \right], \\[2mm]
		\phi = \sum_{\alpha} g_{\alpha},
	\end{gathered}
\end{equation}
entonces puede demostrarse que la ecuaci\'on macrosc\'opica recuperada para $\phi$ satisface:
\begin{equation}
	\dfrac{\partial \phi}{\partial t} + \nabla \cdot (\phi \bm{u}) = \nabla \cdot \left\{ \delta_t(\tau - 0.5) \left[ \dfrac{\partial (\phi\bm{u})}{\partial t} + \nabla \cdot (\phi \bm{uu}) + c_s^2 \nabla \phi\right]\right\}.
	\label{eq:adv_dif_phi_rec}
\end{equation}

En este caso, el coeficiente de difusi\'on recuperado est\'a dado por $k=c_s^2 \delta_t(\tau-0.5)$. A diferencia de la \eq{eq:adv_dif_phi}, la \eq{eq:adv_dif_phi_rec} contiene un t\'ermino de desviaci\'on dado por $\nabla \cdot \{ \delta_t(\tau - 0.5) [ \partial_t (\phi\bm{u}) + \nabla \cdot (\phi \bm{uu}) ]\}$, el cu\'al puede anularse completamente s\'olo en aquellos casos con perfiles de velocidad especiales, como $\bm{u}=const$, $\bm{u}=[u_x(y),0]$, $\bm{u}=[0,u_y(x)]$, etc. 

Ejemplos similares pueden encontrarse dentro de los modelos \pp{} \cite{li_effect_2014}. En ciertas aplicaciones se desea recuperar una ecuaci\'on macrosc\'opica de la forma:
\begin{equation}
	\dfrac{\partial (\rho c_v T)}{\partial t} + \nabla \cdot (\rho c_v T \bm{u}) = \nabla \cdot (\lambda \nabla T),
\end{equation}
donde $\lambda$ es la conductividad t\'ermica y $c_v$ el calor espec\'ifico a volumen constante. En este caso el esquema usual consiste en utilizar una \lbe{} \pp{} est\'andar para la funci\'on de distribuci\'on $f$, asociada a las ecuaciones hidrodin\'amicas, mientras que este esquema tambi\'en puede utilizarse para una segunda distribuci\'on $g$, con $g^{eq}_{\alpha} = c_v T f^{eq}_{\alpha}$ y $\sum g_{\alpha} = \rho c_v T$. Nuevamente, la expansi\'on de Chapman-Enskog demuestra que la ecuaci\'on macrosc\'opica recuperada resulta:
\begin{equation}
	\dfrac{\partial (\rho c_v T)}{\partial t} + \nabla \cdot (\rho c_v T \bm{u}) = \nabla \cdot \left(\lambda \nabla T + \dfrac{\lambda}{p}T\bm{F} \right),
	\label{eq:e_macro_F}
\end{equation}
es decir, aparece un t\'ermino no deseado asociado a la fuerza total de la ecuaci\'on de impulso. Si estos t\'erminos son identificados, como en este caso, entonces pueden ser compensados expl\'icitamente:
\begin{equation}
	g_{\alpha}(\bm{x}+\bm{e}_{\alpha}\delta_t, t+\delta_t) - g_{\alpha}(\bm{x},t) = -\dfrac{1}{\tau} \left[ g_{\alpha}(\bm{x},t) - g^{eq}_{\alpha}(\bm{x},t) \right] + \delta_t C_{\alpha}(\bm{x},t),
\end{equation}
donde $C_{\alpha}$ es un t\'ermino de correcci\'on:
\begin{equation}
	C_{\alpha} = \left( 1-\dfrac{1}{2\tau} \right)\omega_{\alpha}c_v T \dfrac{\bm{e}_{\alpha} \cdot \bm{F}}{c_s^2}.
\end{equation}

En definitiva, los modelos de doble funci\'on de distribuci\'on resultan atractivos, ya que permiten conservar la elegancia y simpleza propia de lattice Boltzmann, mientras que aportan una mayor flexibilidad en el tipo de ecuaciones macrosc\'opicas que pueden recuperarse. Sin embargo, si se desea incrementar la aplicabilidad de estos modelos hacia problemas m\'as complejos, es necesario desarrollar un mecanismo natural para eliminar los t\'erminos no deseados en la ecuaci\'on macrosc\'opica recuperada. Por lo tanto, siguiendo este objetivo primario, en la secci\'on siguiente se introduce un modelo alternativo que permite remediar satisfactoriamente estas limitaciones.


\section{Ecuaci\'on de energ\'ia con operador MRT}
El modelo \pp{} de Li ha sido extensamente validado en la bibliograf\'ia, donde se ha demostrado que puede utilizarse exitosamente en la simulaci\'on de flujos multif\'asicos bidimensionales, con $Re$ moderados y relaci\'on de densidades elevadas (hasta $\rho_l / \rho_g \sim 750$). Adicionalmente, los resultados de la simulaci\'on de un fluido van der Waals mostrados en el \chap{chap:isot} y en \cite{fogliatto_simulation_2019}, sugieren que el modelo es capaz de incorporar adecuadamente fuerzas externas sin afectar el efecto sobre la inconsistencia termodin\'amica. Adem\'as, el an\'alisis de convergencia en malla, que en este contexto se traduce en un incremento de unidades de grilla conservando par\'ametros adimensionales, muestra que es posible alcanzar una precisi\'on adecuada en la reproducci\'on de interfases. Sin embargo, si se desea expandir su uso hacia la simulaci\'on de flujos multif\'asicos con transferencia de calor y cambio de fase, entonces es necesario proveer un nuevo esquema para cuantificar adecuadamente el transporte de energ\'ia.

Dentro del contexto de modelos \pp{}, donde se consideran fluidos cuya presi\'on termodin\'amica se encuentra gobernada por una ecuaci\'on de estado, la ecuaci\'on de energ\'ia macrosc\'opica debe seguir una forma particular. Tomando como base un balance local de entrop\'ia, y despreciando efectos de disipaci\'on viscosa, M\'arkus y H\'azi derivaron una ecuaci\'on macrosc\'opica para la energ\'ia, representada por medio de la temperatura \cite{markus_simulation_2011}:
\begin{equation}
	\dfrac{\partial T}{\partial t} + \bm{u} \cdot \nabla T = \dfrac{1}{\rho c_v} \nabla \cdot(\lambda \nabla T) - \dfrac{T}{\rho c_v} \left( \dfrac{\partial p_{EOS}}{\partial T} \right)_{\rho} \nabla \cdot \bm{u},
	\label{eq:markus_orig}
\end{equation}
donde $\lambda$ es la conductividad t\'ermica y $c_v$ el calor espec\'ifico a volumen constante. La \eq{eq:markus_orig} consiste en una ecuaci\'on de advecci\'on-difusi\'on para $T$, con un t\'ermino de fuente asociado al trabajo de compresi\'on. En las secciones siguientes se considerar\'a que la difusividad t\'ermica $\chi = \lambda/(\rho c_v)$ y el calor espec\'ifico $c_v$ son constantes en cada fase, de modo que la \eq{eq:markus_orig} puede reescribirse como:
\begin{equation}
	\dfrac{\partial T}{\partial t} + \nabla \cdot (\bm{u} T) = \chi \nabla^2 T  + \dfrac{\chi}{\rho} \nabla T \cdot \nabla \rho + T \left[ 1 - \dfrac{1}{\rho c_v} \left( \dfrac{\partial p_{EOS}}{\partial T} \right)_{\rho} \right]
	\label{eq:markus}
\end{equation}

Esta aproximaci\'on, si bien implica una p\'erdida de generalidad, facilita el desarrollo del esquema de lattice Boltzmann. La forma original de la ecuaci\'on puede recuperarse usando t\'erminos de fuente adecuados, como se describe en las secciones siguientes. Para satisfacer esta restricci\'on es posible partir del an\'alisis sobre la simulaci\'on de ecuaciones de advecci\'on-difus\'on llevado a cabo por Huang y colaboradores \cite{huang_modified_2014, huang_numerical_2011, huang_lattice_2015}: la recuperaci\'on de una ecuaci\'on escalar usando una \lbe{} con operador de colisi\'on MRT permite combinar aquellos momentos responsables de los t\'erminos no deseados. Si esta interacci\'on se lleva a cabo de forma inteligente, entonces es posible lograr la supresi\'on de componentes espec\'ificos dentro de las escalas de expansi\'on analizadas. Por otro lado, como se desea recuperar un t\'ermino de fuente, \'este debe incorporarse al esquema sin que se reproduzcan derivadas convectivas de dicha fuente \cite{cheng_introducing_2008}, lo que fuerza a la incorporaci\'on de un t\'ermino de fuente impl\'icito. Combinando estas ideas, una \lbe{} adecuada para recuperar la \eq{eq:markus} resulta:

\begin{equation}
	\begin{aligned}
		\tilde{\bm{g}}(\bm{x} + \bm{e} \delta_t, t+\delta_t) &=
		\tilde{\bm{g}}(\bm{x}, t) - (\bm{M}^{-1}Q\bm{M}) \left[ \tilde{\bm{g}}(\bm{x}, t) - \tilde{\bm{g}}^{eq}(\bm{x}, t) \right] \\
		&  + \dfrac{\delta_t}{2} \hat{\bm{\Gamma}}(\bm{x},t) + \hat{\bm{\Gamma}}(\bm{x} +\bm{e}\delta_t,t + \delta_t),
	\end{aligned}
	\label{eq:g_tilde_2d}	
\end{equation}

\begin{equation}
	\sum_{\alpha} \tilde{g}_{\alpha} = T.
\end{equation}

Siguiendo la notaci\'on usada previamente para las \lbe{} con operador MRT, la componente $\alpha$-\'esima del miembro izquierdo de la \eq{eq:g_tilde_2d} corresponde  a $\tilde{g}_{\alpha}(\bm{x}+\bm{e}_{\alpha}\delta_t)$, y $\bm{M}$ a la matriz de transformaci\'on al espacio de momentos.  En este caso, $\bm{Q}$ corresponde a una matriz de coeficientes relajaci\'on y $\hat{\bm{\Gamma}} = \bm{M}^{-1}\bm{\Gamma}$ a un t\'ermino de fuente en el espacio de poblaciones. Como la \eq{eq:g_tilde_2d} contiene una fuente \'implicita, puede transformarse usando:
\begin{equation}
	g_{\alpha}(\bm{x},t) = \tilde{g}_{\alpha} (\bm{x},t) - \dfrac{\delta_t}{2} \hat{\Gamma}_{\alpha}(\bm{x},t),
	\label{eq:g_tilde_tranf_2d}
\end{equation}
de modo que el momento de orden cero de $\bm{g}$ resulta:
\begin{equation}
	\begin{aligned}
		T &= \sum_{\alpha} \tilde{g}_{\alpha} \\
		  &= \sum_{\alpha} g_{\alpha} + \dfrac{\delta_t}{2} \sum_{\alpha} \tilde{\Gamma}_{\alpha} \\
		  &= \sum_{\alpha} g_{\alpha} + \dfrac{\delta_t}{2} \Gamma_0
	\end{aligned}
	\label{eq:T_macro_2d}
\end{equation}

En la \eq{eq:T_macro_2d} se utiliza que la primera fila de $\bm{M}$ corresponde a $\bm{M}_0 = (1 \cdots 1)$, de forma que $T$ puede obtenerse usando la distribuci\'on $\bm{g}$ y la primera componente de la fuente en el espacio de momentos. La \eq{eq:g_tilde_2d} puede reescribirse usando la \eq{eq:g_tilde_tranf_2d}:
\begin{equation}
	\bm{g}(\bm{x} + \bm{e} \delta_t, t+\delta_t) = \bm{M}^{-1} \left[ \bm{n} - \bm{Q}(\bm{n} - \feq{n}) + \delta_t \left( \bm{I}-\dfrac{\bm{Q}}{2}\right) \bm{\Gamma}\right],
	\label{eq:g_model_2d}
\end{equation}
donde $\bm{n} = \bm{Mg}$ y se omiti\'o la dependencia $(\bm{x},t)$ por claridad. La determinaci\'on de la cuaci\'on macrosc\'opica  recuperada puede hacerse mediante una expansi\'on de Chapman-Enskog, similar a la mostrada en el \red{Cap2}. En primer lugar, es necesario reescribir el t\'ermino izquierdo de la \eq{eq:g_model_2d} empleando un desarrollo de Taylor para cada componente, es decir:
\begin{equation}
	g_{\alpha}(\bm{x}+\bm{e}_{\alpha}\delta_t,t+\delta_t) = \sum_{k=0}^{\infty} \dfrac{1}{k!}\delta_t^k D_{\alpha}^{k} \, g_{\alpha} \sim g_{\alpha} + \delta_t D_{\alpha} g_{\alpha} + \dfrac{1}{2}\delta_t^2 D_{\alpha}^2g_{\alpha},
	\label{eq:taylor_gral}
\end{equation}
donde $D_{\alpha} = \partial_t + \bm{e}_{\alpha} \cdot \nabla=\partial_t + e_{\alpha \beta} \partial_{\beta}$. Para escribir la \eq{eq:taylor_gral} en forma vectorial, 