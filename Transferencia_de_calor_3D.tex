\chapter{Transferencia de calor en flujo multif\'asico tridimensional}

La metodolog\'ia propuesta en el \chap{chap:modelo2D} demostr\'o ser una alternativa robusta para simular transferencia de calor en flujo multif\'asico bidimensional. La elecci\'on de dos \red{\lbe{}} con operador de colisi\'on MRT y definidas sobre un conjunto de velocidades D2Q9 permiti\'o alcanzar una descripci\'on macrosc\'opica precisa de las ecuaciones objetivo, sin la presencia de t\'erminos no deseados en las escalas de expansi\'on analizadas. 

Este modelo bidimensional, preciso y eficiente, puede ser aplicado en problemas reales cuya descripci\'on admite una reducci\'on de dimensiones. Sin embargo, en los problemas t\'ipicos de ebullici\'on esta aproximaci\'on muchas veces no es factible, por lo que es necesario continuar con la extensi\'on del modelo de transferencia de energ\'ia a 3 dimensiones.


\section{Esquema MRT para ecuaciones hidrodin\'amicas}

Los resultados obtenidos con el modelo D2Q9 motivan el desarrollo de nuevas \lbe{} basadas en la familia \pp{}. Por lo tanto, resulta natural buscar una extensi\'on de las ecuaciones a tres dimensiones, pero que tengan como base aquellos modelos desarrollados para grillas bidimensionales.

En el caso de las ecuaciones hidrodin\'amicas, el modelo de Xu y colaboradores \cite{xu_three-dimensional_2015} constituye una extensi\'on del modelo de Li a una grilla D3Q15. Siguiendo una idea similar, este modelo introduce una \lbe{} para una funci\'on de distribuci\'on $f$ con operador de colisi\'on MRT:
\begin{equation}
	\bm{f} (\bm{x} + \bm{e}\delta_t, t+\delta_t) = \bm{M}^{-1}\left[ \bm{m} - \bm{\Lambda}(\bm{m} - \feq{m})  + \delta_t \left( \bm{I} - \dfrac{\bm{\Lambda}}{2} \bar{\bm{S}} \right) + \bm{C}\right],
	\label{eq:lbe_xu}	
\end{equation}
con
\begin{subequations}
	\begin{equation}
		\rho = \sum_{\alpha} f_{\alpha},
	\end{equation}
	\begin{equation}
		\rho \bm{u} = \sum_{\alpha} f_{\alpha} \bm{e}_{\alpha} + \dfrac{\delta_t}{2}\bm{F},
	\end{equation}
\end{subequations}
donde $\bm{F} = \bm{F}_b + \bm{F}_i$ es la fuerza total sobre los nodos, es decir, la combinaci\'on de fuerza boyante y de interaci\'on. Usando la notaci\'on usual, $\bm{m} = \bm{Mf}$ y $\feq{m} = \bm{M}\feq{f}$ corresponden a la proyecci\'on de $\bm{f}$ y $\feq{f}$ en el espacio de momentos respectivamente, de modo que la distribuci\'on de equilibrio en el espacio de momentos resulta:
\begin{equation}
	\begin{split}
		\feq{m} = \rho \left( 1, -1+|\bm{u}|^2, 1-5|\bm{u}|^2, u_x, -\dfrac{7}{3}u_x, u_y, -\dfrac{7}{3}u_y, u_z, -\dfrac{7}{3}u_z, 2u_x^2-u_y^2-u_z^2, \right.\\
		\left.\phantom{\dfrac{1}{1}} u_y^2 - u_z^2, u_xu_y, u_yu_z,u_xu_z,0 \right)^T	.
	\end{split}	
\end{equation}

Por otro lado, la matriz de colisi\'on $\bm{\Lambda}$ corresponde a una matriz diagonal con coeficientes definidos como:
\begin{equation}
	\bm{\Lambda} = \mbox{diag}(\tau_{\rho}^{-1}, \tau_{e}^{-1}, \tau_{\varepsilon}^{-1}, \tau_{j}^{-1}, \tau_{q}^{-1}, \tau_{j}^{-1}, \tau_{q}^{-1}, \tau_{j}^{-1}, \tau_{q}^{-1}, \tau_{\nu}^{-1}, \tau_{\nu}^{-1}, \tau_{\nu}^{-1}, \tau_{\nu}^{-1}, \tau_{\nu}^{-1}, \tau_{xyz}^{-1}),
\end{equation}
mientras que el t\'ermino de fuerza $\bm{S}$ se define en el espacio de momentos mediante:
\begin{equation}
 \bar{\bm{S}} = 
 \left[ 
 	\begin{array}{c} 
 		0	\\
 		2 \bm{u} \cdot \bm{F} + \dfrac{6\sigma |\bm{F}|^2}{\psi^2 \delta_t (\tau_e-0.5)} \\
 		-10 \bm{u} \cdot \bm{F} \\
 		F_x \\
 		-\dfrac{7}{3}F_x \\
 		F_y \\
 		-\dfrac{7}{3}F_y \\
 		F_z \\
 		-\dfrac{7}{3}F_z \\ 		
		4u_xF_x - 2u_yF_y - 2u_zF_z \\
		2u_yF_y - 2u_zF_z \\
		u_xF_y + u_yF_x \\
		u_yF_z + u_zF_y \\
		u_xF_z + u_zF_x \\
		0
 	\end{array} 
 \right]
 \label{eq:s_xu}
\end{equation}

De manera similar a lo que ocurre con la versi\'on de Li \cite{li_lattice_2013}, el par\'ametro $\sigma$ se utiliza para ajustar parcialmente el problema de inconsistencia termodin\'amica, es decir, corregir las densidades de coexistencia recuperadas para cada fase. 

La \eq{eq:lbe_xu} incorpora un t\'ermino adicional $\bm{C}$:
\begin{equation}
 \bm{C} = 
 \left[ 
 	\begin{array}{c} 
 		0	\\
 		\dfrac{4}{5} \tau_{e}^{-1}(R_{xx} + R_{yy} + R_{zz}) \\
 		0 \\
 		0 \\
 		0 \\
 		0 \\
 		0 \\
 		0 \\
 		0 \\ 		
		-\tau_{\nu}^{-1}(2R_{xx} - R_{yy} - R_{zz}) \\
		-\tau_{\nu}^{-1}(R_{yy} - R_{zz}) \\
		-\tau_{\nu}^{-1}R_{xy} \\
		-\tau_{\nu}^{-1}R_{yz} \\
		-\tau_{\nu}^{-1}R_{xz} \\
		0
 	\end{array} 
 \right],
 \label{eq:c_xu}
\end{equation}
donde las componentes $R_{\beta\gamma}$ corresponden al tensor $\bm{R}$ definido como:
\begin{equation}
 \bm{R} = \kappa \dfrac{G}{2} \psi(\bm{x}) \sum_{\alpha} w(|\bm{e}_{\alpha}|^2) \left[ \psi(\bm{x}+\bm{e}_{\alpha}) - \psi(\bm{x}) \right] \bm{e}_{\alpha}\bm{e}_{\alpha}.
	\label{eq:R_xu}
\end{equation}

En este caso, puede demostrarse que la incorporaci\'on del t\'ermino $\bm{C}$ permite  ajustar la tensi\'on superficial recuperada mediante la constante $\kappa$ de la \eq{eq:R_xu}.

\red{Revisar la unificaci\'on de las constantes w}

El an\'alisis de Chapman-Enskog de este esquema muestra que es posible recuperar las siguientes ecuaci\'ones de conservaci\'on de masa e impulso lineal:
\begin{subequations}
	\begin{equation}
		\dfrac{\partial \rho}{\partial t} + \nabla \cdot (\rho \bm{u}) = 0,
	\end{equation}
	\begin{equation}
		\begin{aligned}
		\dfrac{\partial(\rho\bm{u})}{\partial t} + \nabla \cdot (\rho \bm{u} \bm{u})  = &-\nabla \cdot(\rho c_s^2)\bm{I} + \nabla \cdot \bm{\Pi} + \bm{F} - 2G^2 c^4 \sigma \nabla \cdot (|\nabla \psi|^2 \bm{I}) \\
		& -\nabla \cdot \left[ \kappa \dfrac{Gc^4}{6} (\psi \nabla^2 \psi \bm{I} - \psi \nabla \nabla \psi) \right],
		\end{aligned}
	\end{equation}
	\label{eq:xu_macro}
\end{subequations}
donde el tensor de tensiones $\bm{\Pi}$ se define como:
\begin{equation}
	\bm{\Pi} = \rho \nu \left[ \nabla \bm{u} + (\nabla \bm{u})^T \right] + \rho \left( \xi - \dfrac{2}{3}\nu \right) (\nabla \cdot \bm{u})\bm{I},
\end{equation}
y las viscosidades cinem\'atica y \red{de bulk} corresponden a:
\begin{subequations}
	\begin{equation}
		\nu = \dfrac{1}{3} \left( \tau_{\nu} - \dfrac{1}{2}\right) \delta_t,
	\end{equation}
	\begin{equation}
		\xi = \dfrac{2}{9} \left( \tau_{e} - \dfrac{1}{2}\right) \delta_t.
	\end{equation}	
\end{subequations}

Como puede observarse, la \eq{eq:xu_macro} es similar al conjunto de ecuaciones recuperadas por el modelo bidimensional de Li (\eq{eq:li_macro}), salvo por la presencia de t\'erminos dependientes de $\kappa$ que contribuyen a la recuperaci\'on de tensi\'on superficial en la interfase. En particular, si se utiliza le expresi\'on discreta para el tensor de presiones introducida por Shan \cite{shan_pressure_2008}, es posible ver que la forma macorsc\'opica de este tensor finalmente resulta:
\begin{equation}
	\bm{P} = \left[ \rho c_s^2 + \dfrac{G c^2}{2} \psi^2 + \dfrac{G c^4}{12} (1+2\kappa) \psi \nabla^2 \psi + 2 \sigma G^2 c^4 |\nabla \psi|^2\right] \bm{I} + \dfrac{G c^4}{6} (1-\kappa) \psi \nabla \nabla \psi.
\end{equation}







\section{Esquema MRT para ecuaci\'on de energ\'ia}

En el \chap{chap:modelo2D} se mostr\'o que es posible acoplar una segunda \lbe{} a un esquema de la famila \pp{}, y de esta forma poder simular transferencia de calor y cambio de fase en flujos multif\'asicos. En particular, si se elige adecuadamente a la distribuci\'on de equilibrio, matriz de relajaci\'on y t\'ermino fuente, se demostr\'o formalmente que es posible recuperar una soluci\'on macrosc\'opica dada por la \eq{eq:markus}, sin t\'erminos adicionales hasta la escala de expansi\'on analizada. Por otro lado, la metodolog\'ia empleada demostr\'o proveer un camino consistente para el desarrollo de nuevas \lbe{} para la resoluci\'on de ecuaciones de energ\'ia similares, eficientes y flexibles en la reproducci\'on de las propiedades macrosc\'opicas relevantes.

Estas caracter\'isticas positivas observadas en el modelo bidimensional motivan la continuaci\'on de esta l\'inea de desarrollo. De esta manera, la nueva propuesta consiste en emplear una \lbe{} con operador de colisi\'on MRT sobre un conjunto de velocidades D3Q15:
\begin{subequations}
	\begin{equation}
		\bm{g}(\bm{x} + \bm{e} \delta_t, t+\delta_t) = \bm{M}^{-1} \left[ \bm{n} - \bm{Q}(\bm{n} - \feq{n}) + \delta_t \left( \bm{I}-\dfrac{\bm{Q}}{2}\right) \bm{\Gamma}\right],
		\label{eq:g_model_3d}
	\end{equation}	
	\begin{equation}
		T = \sum_{\alpha} g_{\alpha} + \dfrac{\delta_t}{2} \Gamma_0,
	\end{equation}
\end{subequations}

donde las variables del miembro derecho se encuentran definidas en $(\bm{x},t)$. En este caso, $\bm{n} = \bm{Mg}$,  $\feq{n} = \bm{M}\feq{g}$, y $\bm{\Gamma}$ corresponde a un t\'ermino de fuente definido en el espacio de momentos. 

Continuando con las ideas desarrolladas en el \chap{chap:modelo2D}, es posible introducir una distribuci\'on de equilibrio directamente en el espacio de momentos:
\begin{equation}
	\feq{n} = T ( 1, \alpha_1, \alpha_2, \alpha_3u_x, \alpha_4u_x, \alpha_5u_y, \alpha_6u_y, \alpha_7u_z, \alpha_8u_z,0,0,0,0,0,0)^T,
\end{equation}
junto con una matriz de relajaci\'on con coeficientes no nulos en la diagonal, que en este caso corresponde a los elementos $q_{34}$, $q_{56}$ y $q_{78}$.
%\setcounter{MaxMatrixCols}{15}
%\begin{equation}
%	\bm{Q}=
%	\begin{bmatrix}
%	q_0 & 0 & 0 & 0 & 0 & 0 & 0 & 0 & 0 & 0 & 0 & 0 & 0 & 0 & 0 \\
%	0 & q_1 & 0 & 0 & 0 & 0 & 0 & 0 & 0 & 0 & 0 & 0 & 0 & 0 & 0 \\
%	0 & 0 & q_2 & 0 & 0 & 0 & 0 & 0 & 0 & 0 & 0 & 0 & 0 & 0 & 0 \\
%	0 & 0 & 0 & q_3 & q_{34} & 0 & 0 & 0 & 0 & 0 & 0 & 0 & 0 & 0 & 0 \\
%	0 & 0 & 0 & 0 & q_4 & 0 & 0 & 0 & 0 & 0 & 0 & 0 & 0 & 0 & 0 \\
%	0 & 0 & 0 & 0 & 0 & q_5 & q_{56} & 0 & 0 & 0 & 0 & 0 & 0 & 0 & 0 \\
%	0 & 0 & 0 & 0 & 0 & 0 & q_6 & 0 & 0 & 0 & 0 & 0 & 0 & 0 & 0 \\
%	0 & 0 & 0 & 0 & 0 & 0 & 0 & q_7 & q_{78} & 0 & 0 & 0 & 0 & 0 & 0 \\
%	0 & 0 & 0 & 0 & 0 & 0 & 0 & 0 & q_8 & 0 & 0 & 0 & 0 & 0 & 0 \\
%	0 & 0 & 0 & 0 & 0 & 0 & 0 & 0 & 0 & q_{9} & 0 & 0 & 0 & 0 & 0 \\
%	0 & 0 & 0 & 0 & 0 & 0 & 0 & 0 & 0 & 0 & q_{10} & 0 & 0 & 0 & 0 \\
%	0 & 0 & 0 & 0 & 0 & 0 & 0 & 0 & 0 & 0 & 0 & q_{11} & 0 & 0 & 0 \\
%	0 & 0 & 0 & 0 & 0 & 0 & 0 & 0 & 0 & 0 & 0 & 0 & q_{12} & 0 & 0 \\
%	0 & 0 & 0 & 0 & 0 & 0 & 0 & 0 & 0 & 0 & 0 & 0 & 0 & q_{13} & 0 \\
%	0 & 0 & 0 & 0 & 0 & 0 & 0 & 0 & 0 & 0 & 0 & 0 & 0 & 0 & q_{14}
%	\end{bmatrix}
%\end{equation}

La expansi\'on en serie de Taylor de la \eq{eq:g_model_3d} lleva a una expresi\'on similar a la \eq{eq:model_2d_momento}:
\begin{equation}
	\dcm \bm{n} + \dfrac{\delta_t}{2} \dcm^2 \bm{n} = -\dfrac{1}{\delta_t}\bm{Q}(\bm{n} - \feq{n}) + \delta_t \left( \bm{I}-\dfrac{\bm{Q}}{2}\right) \bm{\Gamma},
	\label{eq:model_3d_momento}
\end{equation}
y, por lo tanto, a la misma descomposici\'on en escalas de $\varepsilon$:
\begin{subequations}
	\begin{align}
		\varepsilon^0: && \bc{n}{0} = \feq{n} \label{eq:eps_0_3d}\\
		\varepsilon^1: && \dcmuno \bc{n}{0} - \bc{\Gamma}{1} = -\dfrac{1}{\delta_t} \bm{Q} \left( \bc{n}{1} + \dfrac{\delta_t}{2} \bc{\Gamma}{1} \right)  \label{eq:eps_1_3d}\\
		\varepsilon^2: && \dcmuno \left( \bm{I}-\dfrac{\bm{Q}}{2}\right) \left( \bc{n}{1} + \dfrac{\delta_t}{2} \bc{\Gamma}{1} \right) + \dfrac{\partial \bc{n}{0}}{\partial t_2}  =  -\dfrac{1}{\delta_t} \bm{Q} \bc{n}{2}. \label{eq:eps_2_3d}
	\end{align}
\end{subequations}

En una grilla D3Q15, sin embargo, las matrices $\hat{\bm{E}}_{\beta}$, con $\beta=x,y,z$, quedan definidas como:
\setcounter{MaxMatrixCols}{15}
\begin{equation}
	\hat{\bm{E}}_{x}=
	\begin{bmatrix}
	0 & 0 & 0 & 1 & 0 & 0 & 0 & 0 & 0 & 0 & 0 & 0 & 0 & 0 & 0 \\
	0 & 0 & 0 & 3/5 & 2/5 & 0 & 0 & 0 & 0 & 0 & 0 & 0 & 0 & 0 & 0 \\
	0 & 0 & 0 & 0 & 1 & 0 & 0 & 0 & 0 & 0 & 0 & 0 & 0 & 0 & 0 \\
	2/3 & 1/3 & 0 & 0 & 0 & 0 & 0 & 0 & 0 & 1/3 & 0 & 0 & 0 & 0 & 0 \\
	0 & 8/9 & 1/9 & 0 & 0 & 0 & 0 & 0 & 0 & -4/3 & 0 & 0 & 0 & 0 & 0 \\
	0 & 0 & 0 & 0 & 0 & 0 & 0 & 0 & 0 & 0 & 0 & 1 & 0 & 0 & 0 \\
	0 & 0 & 0 & 0 & 0 & 0 & 0 & 0 & 0 & 0 & 0 & 1 & 0 & 0 & 0 \\
	0 & 0 & 0 & 0 & 0 & 0 & 0 & 0 & 0 & 0 & 0 & 0 & 0 & 1 & 0 \\
	0 & 0 & 0 & 0 & 0 & 0 & 0 & 0 & 0 & 0 & 0 & 0 & 0 & 1 & 0 \\
	0 & 0 & 0 & 2/5 & -2/5 & 0 & 0 & 0 & 0 & 0 & 0 & 0 & 0 & 0 & 0 \\
	0 & 0 & 0 & 0 & 0 & 0 & 0 & 0 & 0 & 0 & 0 & 0 & 0 & 0 & 0 \\
	0 & 0 & 0 & 0 & 0 & 4/5 & 1/5 & 0 & 0 & 0 & 0 & 0 & 0 & 0 & 0 \\
	0 & 0 & 0 & 0 & 0 & 0 & 0 & 0 & 0 & 0 & 0 & 0 & 0 & 0 & 1 \\
	0 & 0 & 0 & 0 & 0 & 0 & 0 & 4/5 & 1/5 & 0 & 0 & 0 & 0 & 0 & 0 \\
	0 & 0 & 0 & 0 & 0 & 0 & 0 & 0 & 0 & 0 & 0 & 0 & 1 & 0 & 0 \\
	\end{bmatrix},
\end{equation} 

\begin{equation}
	\hat{\bm{E}}_{y}=
	\begin{bmatrix}
	0 & 0 & 0 & 0 & 1 & 0 & 0 & 0 & 0 & 0 & 0 & 0 & 0 & 0 & 0 \\
	0 & 0 & 0 & 0 & 3/5 & 2/5 & 0 & 0 & 0 & 0 & 0 & 0 & 0 & 0 & 0 \\
	0 & 0 & 0 & 0 & 0 & 0 & 1 & 0 & 0 & 0 & 0 & 0 & 0 & 0 & 0 \\
	0 & 0 & 0 & 0 & 0 & 0 & 0 & 0 & 0 & 0 & 0 & 1 & 0 & 0 & 0 \\
	0 & 0 & 0 & 0 & 0 & 0 & 0 & 0 & 0 & 0 & 0 & 1 & 0 & 0 & 0 \\
	2/3 & 1/3 & 0 & 0 & 0 & 0 & 0 & 0 & 0 & -1/6 & 1/2 & 0 & 0 & 0 & 0 \\
	0 & 8/9 & 1/9 & 0 & 0 & 0 & 0 & 0 & 0 & 2/3 & -2 & 0 & 0 & 0 & 0 \\
	0 & 0 & 0 & 0 & 0 & 0 & 0 & 0 & 0 & 0 & 0 & 0 & 0 & 1 & 0 \\
	0 & 0 & 0 & 0 & 0 & 0 & 0 & 0 & 0 & 0 & 0 & 0 & 0 & 1 & 0 \\
	0 & 0 & 0 & 0 & -1/5 & 1/5 & 0 & 0 & 0 & 0 & 0 & 0 & 0 & 0 & 0 \\
	0 & 0 & 0 & 0 & 1/5 & -1/5 & 0 & 0 & 0 & 0 & 0 & 0 & 0 & 0 & 0 \\
	0 & 0 & 0 & 4/5 & 1/5 & 0 & 0 & 0 & 0 & 0 & 0 & 0 & 0 & 0 & 0 \\
	0 & 0 & 0 & 0 & 0 & 0 & 0 & 4/5 & 1/5 & 0 & 0 & 0 & 0 & 0 & 0 \\
	0 & 0 & 0 & 0 & 0 & 0 & 0 & 0 & 0 & 0 & 0 & 0 & 0 & 0 & 1 \\
	0 & 0 & 0 & 0 & 0 & 0 & 0 & 0 & 0 & 0 & 0 & 0 & 0 & 1 & 0 \\
	\end{bmatrix}
\end{equation} 

y 

\begin{equation}
	\hat{\bm{E}}_{z}=
	\begin{bmatrix}
	0 & 0 & 0 & 0 & 0 & 0 & 0 & 1 & 0 & 0 & 0 & 0 & 0 & 0 & 0 \\
	0 & 0 & 0 & 0 & 0 & 0 & 0 & 3/5 & 2/5 & 0 & 0 & 0 & 0 & 0 & 0 \\
	0 & 0 & 0 & 0 & 0 & 0 & 0 & 0 & 1 & 0 & 0 & 0 & 0 & 0 & 0 \\
	0 & 0 & 0 & 0 & 0 & 0 & 0 & 0 & 0 & 0 & 0 & 0 & 0 & 1 & 0 \\
	0 & 0 & 0 & 0 & 0 & 0 & 0 & 0 & 0 & 0 & 0 & 0 & 0 & 1 & 0 \\
	0 & 0 & 0 & 0 & 0 & 0 & 0 & 0 & 0 & 0 & 0 & 0 & 1 & 0 & 0 \\
	0 & 0 & 0 & 0 & 0 & 0 & 0 & 0 & 0 & 0 & 0 & 0 & 1 & 0 & 0 \\
	2/3 & 1/3 & 0 & 0 & 0 & 0 & 0 & 0 & 0 & -1/6 & -1/2 & 0 & 0 & 0 & 0 \\
	0 & 8/9 & 1/9 & 0 & 0 & 0 & 0 & 0 & 0 & 2/3 & 2 & 0 & 0 & 0 & 0 \\
	0 & 0 & 0 & 0 & 0 & 0 & 0 & -1/5 & 1/5 & 0 & 0 & 0 & 0 & 0 & 0 \\
	0 & 0 & 0 & 0 & 0 & 0 & 0 & -1/5 & 1/5 & 0 & 0 & 0 & 0 & 0 & 0 \\
	0 & 0 & 0 & 0 & 0 & 0 & 0 & 0 & 0 & 0 & 0 & 0 & 0 & 0 & 1 \\
	0 & 0 & 0 & 0 & 0 & 4/5 & 1/5 & 0 & 0 & 0 & 0 & 0 & 0 & 0 & 0 \\
	0 & 0 & 0 & 4/5 & 1/5 & 0 & 0 & 0 & 0 & 0 & 0 & 0 & 0 & 0 & 0 \\
	0 & 0 & 0 & 0 & 0 & 0 & 0 & 0 & 0 & 0 & 0 & 1 & 0 & 0 & 0 \\
	\end{bmatrix}.
\end{equation} 

